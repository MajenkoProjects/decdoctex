\chapter{Systems Introduction}
\section{The VAXstation 2000 and MicroVAX 2000 Systems}
\pagestyle{main}
\pagenumbering{arabic}
\setcounter{page}{1}


The VAXstation 2000 and MicroVAX 2000 systems are mechanically
identical. Both come in the same style box, both use the same drives,
and both use the same mass storage expansion boxes. Also, both use
the same diagnostic tools for troubleshooting and repair. Once familiar
with troubleshooting one system, you'll be able to troubleshoot the other
if necessary. One major difference is the VAXstation 2000 is a single-user
system and the MicroVAX 2000 is a multiuser system. Another difference
is the VAXstation 2000 uses a video monitor while the MicroVAX 2000 uses
video terminals.

Both the VAXstation 2000 and the MicroVAX 2000 have three main pieces
of hardware. They are the System box, the hard disk expansion box, and
the tape drive expansion box. The System box can have a half-height RX33
floppy disk drive, a half-height RD32 hard disk drive, or both the RX33 and
the RD32. A full-height RD53 hard disk drive can be substituted for the 
half-height drives in the System box. The hard disk expansion box comes with a
full-height RD53 hard disk drive. The tape drive expansion box comes with
a TK50 tape drive.

\newpage

\hyperref[figure:1-1]{Figure 1-1} shows the front of the VAXstation 2000 and MicroVAX 2000
systems. There are three ways to differentiate between the two systems:
the medallion next to the power switch on the front, the DEC423 converter
on the back, or the system jumper position on the system module inside
the box.

\fig{MA-0063-87}{Front View of the VAXstation 2000 and MicroVAX 2000 Systems}
\label{figure:1-1}

\newpage
\hyperref[figure:1-2]{Figure 1-2} shows the rear view of the VAXstation 2000 and labels each
connector. A modem or a terminal can be connected to the 25-pin 
communication port. A VR260 monochrome monitor can be connected to the
15-pin video port. A printer can be connected to the 9-pin printer port. The
ThinWire Ethernet port Supports IEEE 802.3 (Standard Ethernet) network
communications connections over the ThinWire Ethernet cable.

\fig{MA-0132-87}{Rear View of the VAXstation 2000 System}
\label{figure:1-2}

\newpage


\hyperref[figure:1-3]{Figure 1-3} shows the rear view of the MicroVAX 2000 and labels each 
connector. Like the VAXstation 2000, the MicroVAX 2000 supports a modem
or a terminal on the 25-pin communication port. The MicroVAX 2000 
supports the DECconnect strategy which uses the modified modular jack (MMJ)
6-conductor telephone type cable (DEC423 asynchronous protocol) for 
connection to the terminals. The DEC423 Converter changes the 15-pin video
port and the 9-pin printer port (RS232 protocol) to three MMJ communication 
ports. Port 1 on the DEC423 Converter is reserved for the console
terminal. Ports 2 and 3 can have either a terminal or a printer attached to
them. The operating system Software configures each port for either a terminal 
or a printer. The ThinWire Ethernet port supports IEEE 802.3 (Standard
Ethernet) network communications connections over the ThinWire Ethernet
cable.

\fig{MA-0134-87}{Rear View of the MicroVAX 2000 System}
\label{figure:1-3}

\section{Mass Storage Expansion Box for Both Systems}

Additional mass storage devices are contained in expansion boxes that look
very similar to the system box. \hyperref[figure:1-4]{Figures 1-4} and \hyperref[figure:1-5]{1-5} show the front view of
the expansion boxes.

The hard disk expansion box contains an RD53 or RD54 hard disk drive.
The tape drive expansion box contains a TK50 tape drive and a controller
board. Each expansion box contains a power supply, a resistor load board
(to regulate the power supply), and the drive.

\fig{MA-0065-87}{Front View of the Hard Disk Expansion Box}
\label{figure:1-4}

\newpage

\fig{MA-0064-87}{Front View of the Tape Drive Expansion Box}
\label{figure:1-5}

\newpage

Both expansion boxes connect to the system box through an expansion
adapter that attaches to the bottom of the system box. The expansion
adapter has three connectors on the back labeled ports A, B, and C. Port
A connects the tape expansion box to the system. Port B connects the hard
disk expansion box to the system. Port C is reserved for future options. 
\hyperref[figure:1-6]{Figure 1-6} shows the back of a MicroVAX 2000 system box with an expansion
adapter.

\fig{MA-0135-87}{System Box with Expansion Adapter}
\label{figure:1-6}

\newpage

\section{Options}
\subsection{Internal Memory Options}

Two additional memory modules are available for both systems. One is a 
2-megabyte memory module and the other is a 4-megabyte memory module.
The memory module is located in the system box and is connected directly
to the system module.

\subsection{ThinWire Ethernet Option on MicroVAX 2000}

ThinWire Ethernet is an option on the MicroVAX 2000. It comes standard
on the VAXstation 2000. It adds the capability of connecting the system to
the DECnet through the ThinWire Ethernet network. The option consists
of a network interconnect module that is located in the system box and is
connected to the system module through two 40-conductor cables.

\section{FRU Locations}

\hyperref[figure:1-7]{Figure 1-7} shows the locations of the FRUs in the system box. Figure 1-8
shows the locations of the FRUs in the expansion boxes.

\fig{MA-0150-87}{FRU Locations in the System Box}
\label{figure:1-7}

\newpage

\fig{MA-0133-87}{FRU Locations in the Expansion Boxes}
\label{figure:1-7}

