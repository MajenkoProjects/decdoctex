\documentclass{decsectional}
\product{DEC Documentation}
\title{ReadMe}
\pubmonth{March}
\pubyear{2025}
\author{The DEC Documentation Rebuild Project}
\address{GitHub}

\begin{document}
\toc
\newpage

\uchapter{Preface}
\pagestyle{preface}

Welcome to the DEC Documentation Rebuild project.  This project, driven by the community, aims to
recreate as much of the documentation for vintage Digital Equipment Corporation (DEC) computer 
equipment and software as possible.

The reason for this is twofold:

\begin{enumerate}

\item 	Much of the documentation is in a poor state. It can be hard to make out some parts of it due to bad scanning,
		or generally poor source material.

\item	The existing documentation can be very hard to work with. Scanned PDFs, even with advanced OCR, are hard to
		search and navigate. Rebuilding the documentation gives us the chance to correct that with embedded hyperlinks,
		clean text, and a generally more usable experience.

\end{enumerate}

\chapter{How we are doing it}
\pagestyle{main}

It is highly likely (backed up by some evidence from the documentation itself\footnote{For example using a space separated console prompt such as {\tt > > >}
to combat \LaTeX\ conversion of {>}{>} into >> }) the original documentation is written in \LaTeX. In order to try and maintain as much of the original layout and pagination as closely
as possible to the original \LaTeX\ has again been chosen as the method to re-create the documentation. This also lets us separate out the style from the content
making it much faster, once suitable document classes have been created, to recreate documents with the minimum of effort.

\chapter{Contributing}

You'd like to contribute? Fantastic! We are always looking for more volunteers to help recreate more documents. Just fork this repository and get
writing. Most of the groundwork has been done for you in the form of some handy document classes (dec.cls and decsectional.cls) that implement
reasonably accurately\footnote{Apart from the fonts which we have tried to find reasonable matches for in the standard \LaTeX\ font library, but
has proved almost impossibe - and we don't want to have to use third party fonts to complicate matters.}. By all means take a look at one of the
existing \LaTeX\ files for an idea of how to go about implementing the document.

We only have a few stipulations when it comes to style, both of content and general working:

\begin{itemize}

\item	The hyperref package is automatically included in the base dec.cls file. Please use hyperlinks and hyperrefs within the document
		to link to sections, figures and tables where they are mentioned in the text (see below for helper functions for these). 
		Also please use the {\tt \\pdf\{...\}\ } command to
		wrap any references to other DEC documents. This just creates a href to a PDF document in the same directory at the moment though
		that may be subject to change in the future.

\item	Your document should be named by the order number of the document (for example EK-VAXAC-OM-003.tex) with any sub-parts being named the same but
		with a hyphenated suffix (for example EK-VAXAC-OM-003-ch1.tex).

\item	In DEC documentation all figures have a reference number associated with them which denotes the author, their image sequence number, and the
		year of production. When you cut out an image from the original scanned PDF please include this reference number. Name the image file with
		this reference number and place it in the \textbf{fig} folder, then use the {\tt \\fig\{ref\}\{caption\}\ } command to reference it within
		your document.

\item	Title page images should be stored in the \textbf{titles} folder and named after the order number of the document.

\end{itemize}

When transcribing you should attempt to match the layout and pagination of the original document as possible. This is chiefly so that someone
who is referencing the original scanned PDF and someone who is referencing the rebuilt PDF both get the same page numbers for the same information and
can collaborate more seamlessly. Some bleed of paragraphs from page to page is fine, but tables, figures, and sections should be on the same pages
as the original where possible.

\chapter{Helper Functions}

We have a number of handy helper functions to aid in keeping the layout of the document as close to the original
as possible without you having to think too hard about how to do it.

They are included as (currently) two class files, \texttt{dec.cls} and \texttt{decsectional.cls}. The former is the master class
which is geared towards simpler non-numbered (single chapter) documents. The latter extends the master class to allow
creation of longer chapter based documents.

\section{Figures}

There are two figure helper functions, \texttt{fig} and \texttt{ttfig}.  The first of these is used to include a figure
into the document at the current location.

\begin{verbatim}
\fig[Scale]{ImageRefCode}{Caption For This Figure}
\end{verbatim}

The \texttt{Scale} parameter is optional and sets the width of the image as a percentage (0.0 - 1.0) of the page width.
The \texttt{ImageRefCode} is the ID code (XX-NNNN-YY) of an image within the fig directory, and the caption is placed
above the image and included in the list of figures in the contents section.

The \texttt{ttfig} is a little different in that it defines a new environment which is used for creating
text-based (ASCII art, console display, etc) figures.


\begin{verbatim}
\begin{ttfig}{This is the caption}
 _____ _                      
|  ___(_) __ _ _   _ _ __ ___ 
| |_  | |/ _` | | | | '__/ _ \
|  _| | | (_| | |_| | | |  __/
|_|   |_|\__, |\__,_|_|  \___|
         |___/ 
\end{ttfig}
\end{verbatim}

Result:

\begin{ttfig}{This is the caption}
 _____ _
|  ___(_) __ _ _   _ _ __ ___
| |_  | |/ _` | | | | '__/ _ \
|  _| | | (_| | |_| | | |  __/
|_|   |_|\__, |\__,_|_|  \___|
         |___/
\end{ttfig}

A DEC-style label is automatically created for every figure (figure:F or figure:C-F) for hyperlinks to jump to.

\section{Tables}

Tables are internally handled by the \texttt{tabularx} package, but are wrapped in extra code to handle DEC style
labels and captions. The main table environment is:

\begin{verbatim}
\begin{tbl}{Caption Here}{Spec}
... content ...
\end{tbl}
\end{verbatim}

The \texttt{Spec} is a normal tabularx column set specification describing the columns in the table.  A top and
bottom horizontal line are automatically added, so just add the headings, another hline, and then the table body.
For example:

\begin{verbatim}
\begin{tbl}{A Sample Table}{c c}
\textbf{First column} & \textbf{Second column} \\
\hline
This is something & This is something else \\
This is more & This is even more \\
\end{tbl}
\end{verbatim}

The result:

\begin{tbl}{A Sample Table}{c c}
\textbf{First column} & \textbf{Second column} \\
\hline
This is something & This is something else \\
This is more & This is even more \\
\end{tbl}

If a table is too long to fit on one page you can finish the table early, then re-start it on the next page
using the \texttt{tblcont} environment. This is exactly the same as the \texttt{tbl} environment except the
word (Cont.) is added to the caption numbering, and the table is not included in the list of tables in the
TOC.

\begin{verbatim}
\begin{tblcont}{A Sample Table}{c c}
\textbf{First column} & \textbf{Second column} \\
\hline
This is exta & This bit wouldn't fit in the previous table.\\
\end{tblcont}
\end{verbatim}

\begin{tblcont}{A Sample Table}{c c}
\textbf{First column} & \textbf{Second column} \\
\hline
This is exta & This bit wouldn't fit in the previous table.\\
\end{tblcont}

\section{Chapters and sections}

As well as the normal chapter and section (both starred and unstarred variant) commands we have u-prefixed 
variants which serve as a half-way house between the starred and unstarred variants. Like the starred variants
they are unnumbered, but like the unstarred variants they are included in the TOC. This allows for unnumbered
documents to be created yet still have a functional TOC with minimum fuss.



\section{References}

Creating links within the document is made easier with the use of a few reference helper functions: \texttt{figref} and
\texttt{tableref}.  Both just take a DEC-style figure or table reference number (for example 2-5) and format the
name of the link for you automatically.

There is also a \texttt{pdf} helper function which just takes a DEC order number and links to the PDF externally.







\chapter{License}

These documents are provided with no warranty as regards their accuracy whatsoever. The document class files are provided under the CC-BY 4.0 license
for you to use and adapt for your own purposes as you see fit. We hope you find them useful. The document content and images remain \copyright Digital
Equipment Corporation or the current owner of their trademarks and copyrights (either HPE or VSI at the moment) and are provided and used as an educational
resource for archival and learning purposes.




\end{document}
