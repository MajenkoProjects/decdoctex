\documentclass{dec}
\product{VAXstation 2000}
\title{Hardware Option Guide for the 4-Plane Graphics Coprocessor}
\ordernumber{EK-VAXAA-4P-001}
\author{digital equipment corporation}
\address{maynard, massachusetts}
\pubmonth{November}
\pubyear{1987}
\titlepicture{titles/EK-VAXAA-4P-001}

\begin{document}
\maketitle

\toc

\newpage
\uchapter{Preface}
\thispagestyle{preface}
The VAXstation 2000 4-plane graphics coprocessor provides a 4-plane video
Subsystem for the VAXstation 2000 System. This guide is intended for
users who are upgrading their VAXstation 2000 System to support a 4-plane
graphics System with a monochrome monitor. If you have a color monitor,
see the documentation packaged with the color monitor.

This guide describes the 4-plane graphics coprocessor and gives some
diagnostic Information on this Option. It also describes how to install the color
cable that is needed to use a monochrome monitor for gray-scale display.

The VAXstation 2000 4-plane graphics coprocessor is installed by a DIGITAL
service representative.

\subsection*{Recommended Reading Path}
Read the VAXstation 2000 hardware and operating system Software documentation in the following order:
\begin{enumerate}
	\item Hardware Installation Information
	\begin{enumerate}
		\item VAXstation 2000 Hardware Installation Guide (Order No. EK-VAXAC-IN-003.)
		\item To install a monochrome monitor, see this guide, the VAXstation 2000 Hardware Option Guide for the 4-Plane Graphics Coprocessor (Order No.  EK-VAXAA-4P-001.)
		\item Workstations and MicroVAX 2000 Network Guide (Order No EK-NETAB-UG)
	\end{enumerate}
	\item Operating System Information
	\begin{enumerate}
		\item VAXstation 2000 Owner's Manual, Chapter 1 (Operating the VAXstation 2000) (Order No. EK-VAXAC-OM-003)
		\item MicroVMS VAXstation 2000/MicroVAX 2000 Installation Guide (Order No. AA-JR84A-TN)

				or

				ULTRIX-32W Basic Installation Guide for the VAXstation 2000 (Order No. AA-KU45A-TE)
	\end{enumerate}
	\item Reference Information
	\begin{enumerate}
		\item VAXstation 2000 Owner's Manual (Order No. EK-VAXAC-OM-003)
		\item MicroVMS VAXstation 2000/MicroVAX 2000 Operations Guide (Order No. AA-JR83A-TN)
	\end{enumerate}
\end{enumerate}

\subsection*{Special Notices}
The following notices appear throughout this guide:

\begin{itemize}
	\item Notes - Contain general or supplemental information about a topic.
	\item Cautions - COntain information to prevent damage to equipment.
\end{itemize}

\subsection*{Conventions Used in This Guide}

\begin{tabular}{m{3cm} m{7.5cm}}
\hline
\textbf{Convention} & \textbf{Meaning} \\
\hline
\textbf{Bold} & Notes, cautions and warnings appear in bold face. User input is also in bold.\\
\hline
\end{tabular}

\newpage
\setcounter{page}{1}
\pagenumbering{arabic}
\pagestyle{main}

\usection{Setting Up the Color or Gray-Scale 4-Plane Graphics Option}

If you have a monochrome monitor, use the instructions in this guide to
install a color cable (packaged in the Upgrade kit) on the monitor.
 If you ordered a new monitor with your Upgrade kit, first set it up as described in
the monitor installation documentation, packaged with the monitor.
 Then, after the Option is installed, follow the instructions in this guide.

The VAXstation 2000 4-plane graphics coprocessor provides a 4-plane video
Subsystem for the VAXstation 2000 System. This raster scan video Subsystem
provides a resolution of 1024 by 864 pixels, and displays up to 16 gray-scales
or 16 colors from a palette of 4096.

A bitmap display provides variable character size or positioning and inherent
graphics capability. The display provides about 850,000 pixels refreshed on
the screen at 60 Hertz, thus preventing the flicker and smear of an interlaced
display.

\usection{Installing the Video Cable Assembly on a Monochrome Monitor}

If you have a VR260 or VR150 monochrome monitor with a 4-plane graphics
coprocessor to display gray-scale on your screen, you will need to install a
color video cable assembly on your monitor.

First, locate the video cable assembly shown in Figure 1 (part no. BC19S).
\fig{MLO-1263-87}{Color Video Cable Assembly, BC19S}

The following sections describe the installation process for the VR260 and
VR150 respectively.

\usubsection{VR260 Monochrome Monitor with 4-Plane Graphics Coprocessor}

\begin{enumerate}
\item Make sure the monitor's power supply is turned off.
\item Insert only the G video cable connector of the color video cable assembly
(BC19S) into the socket on the back of the monitor (Figure 2).
\item Turn the connector clockwise to fasten.

\textbf{Note: The R and B connectors will hang freely — do not connect
them to anything.}

\fig{MLO-1269-87}{Connecting G Cable Connector, VR260 Monochrome Monitor}

\item Locate the shorter of the two knobs that come with the color video cable.
\item Insert the shorter knob into the color video cable assembly.
\item Turn the knob on the video cable assembly clockwise to secure it to the
monitor (Figure 3).

\fig{MLO-1270-87}{Video Cable Assembly Knob}

\end{enumerate}


\usubsection{VR150 Monochrome Monitor with 4-Plane Graphics Coprocessor}

\begin{enumerate}
\item Make sure the monitor's power supply is turned off.
\item Locate the shorter of the two knobs that come with the color video cable.
\item Insert the shorter knob into the color video cable assembly.
\item Turn the knob on the video cable assembly clockwise to secure it to the monitor (Figure 3).
\item Screw the video cable assembly into the cable mounting plate (Figure 4).
\fig{MLO-1306-87}{Mounting Video Cable Assembly on VR150 Monitor, Color Cable}
\textbf{Caution: Failure to secure the video cable assembly to the cable
mounting plate may strain the cable and may also cause radio
emissions in excess of FCC guidelines.}
\item Connect only the G video cable to the video connector on the rear panel
(Figure 5).
\item Push the cable in and turn the connector clockwise to lock it in place.
\end{enumerate}
\textbf{Note: The R and B connectors will hang freely — do not connect them
to anything.}

\fig{MLO-1279-87}{Connecting G Cable Connector, VR150 Monitor}
\usection{Connecting the Monitor Cable to the System Unit}

Use the color video cable assembly (part no. BC19S) to connect your monitor
to the System unit.

Connect the monitor cable to the System unit as follows:

\textbf{Caution: When installing the cables into the rear of the VAXstation 2000
system unit, do not place the System unit on its front, as it may damage
the drive door.}

\begin{enumerate}
\item Make sure the power supplies for the monitor and System unit are turned
off.
\item  Lift up the cable-restraining bar on the rear of the VAXstation 2000 System
unit (Figure 6).
\fig{MLO-2116-87}{Connecting the Monitor Cable to the System Unit}
\textbf{Note: If you plan to connect the printer and modern cables, leave the
cable-restraining bar up until you have connected those options.}
\item Connect the end of the BC19S video cable to the video port on the rear
of the VAXstation 2000 System unit.
 The video port is identified by a
video icon. Insert the cable with the video-icon side of the monitor cable
facing you.
\item Tighten the thumbscrews on the monitor video cable by turning them
clockwise with your hngers (Figure 7).
\end{enumerate}
\textbf{Caution: Before connecting the monitor cable to the System, make sure
the system's power supply is turned off.}
\fig{MLO-2117-87}{Tightening the Thumbscrews on the Monitor Cable}


\usection{Testing Your System}
The VAXstation 2000 diagnostic tests are stored in the VAXstation 2000 read-
only memory. After every power up, the VAXstation 2000 automatically Starts
the System tests, which check the components of your System. The following
section describes some of the diagnostic tests available on the VAXstation
2000.
 For more information, see the \textit{VAXstation 2000 Hardware Installation
Guide.}

If you are going to perform diagnostic tests, make sure that both the monitor
and System unit power supplies are turned on.

\usubsection{A Note on Diagnostics}
In diagnostics, a Single question mark precedes additional Status information.
An example of Status information is:

{\tt ? 4 00D0  0009.0213}

This example indicates that there is a 4-plane coprocessor present in the
System with a monochrome monitor and a color cable.

If the 4-plane module has a fatal error, the monitor screen will remain blank.

\textbf{Note: If your monitor does not display any information on the screen after
the power-up test sequence, you may have a failure in your video board
and/or other failures.  Check the LEDs on your keyboard for secondary
power-up error messages. If any of the LEDs in Table 1 are lit, call your
service representative.}

\begin{table}[H]
\label{table:1}
\caption{Power-Up Error Messages}
\begin{tabularx}{\textwidth}{l l}
\hline
\textbf{If} & \textbf{Then there is a}\\
\hline
The Hold Screen keyboard LED is lit & System failure\\
The Lock keyboard LED is lit & Video board failure \\
The Compose keyboard LED is lit & Communication option failure\\
\hline
\end{tabularx}
\end{table}

\usubsection{TEST 50 Mnemonics}
When you run TEST 50, information about the system is displayed using the mnemonics listed in Table 2.

\begin{table}[H]
\label{table:2}
\caption{Definitions of TEST 50 Mnemonics}
\begin{tabularx}{\textwidth}{l l}
\hline
\textbf{Mnemonic} & \textbf{Device}\\
\hline
CLK & System clock \\
4PLN & 4-plane graphics coprocessor \\
DZ & Serial line controller \\
FP & Floating point \\
HDC & Disk controller \\
IT & Interval timer \\
MEM & Memory \\
MM & Memory management \\
MONO & Monochrome video circuits on system module \\
NVR & Nonvolatile RAM \\
SYS & Interrupt controller and Ethernet ID ROM \\
TPC & Tape controller \\
\hline
\end{tabularx}
\end{table}

The HDC mnemonic represents the disk Controller. The first group of numbers
after the HDC line in the screen display teils you what type of device is
installed.

\usubsection{Monitor Screen Pattern Tests}

Screen pattern tests are useful if you suspect a malfunction in your monitor.
Record what you see and report it to your Service representative.

\textbf{Note: Check your Software documentation for shutdown procedures before halting your System.}

Put the VAXstation 2000 in console mode by pressing the halt button on the
rear of the System. Then run any of the tests listed. When you have finished,
type \textbf{BOOT} (or \textbf{BOOT} followed by the name of the device that contains
operating System Software) at the console prompt and press Return to return
to normal operation.

\newpage

\usubsubsection{Color Monitor}
\begin{enumerate}
\item Screen of Es Pattern

Type \textbf{TEST 81} and press Return.

The monitor displays a full screen of Es. Press Return to stop the display and return to the console prompt.

\item White Screen

Type \textbf{TEST 82} and press Return.

The monitor displays a white screen. Press Return to stop the display and return to the console prompt.

\item Eight Color Bars

Type \textbf{TEST 87} and press Return.

The monitor display eight color bars (Figure 8).
\end{enumerate}

\fig{MLO-1312-87}{Color Bars}

\usubsubsection{Gray-Scale Patterns}

If you are using gray-scale on your monochrome monitor and suspect a
malfunction in your monitor, use the following test for gray-scale patterns:
\begin{itemize}
\item Eight Gray-Scale Bars

Type \textbf{TEST 88} and press Return.

The monitor displays eight gray-scale bars from black to white.
 Press Return to stop the display and return the console mode prompt.
\end{itemize}

\usubsection{Restarting the System After Running Tests}

While you are running any of the tests or procedures in this chapter, you are
in console mode. To resume normal Operation of the VAXstation 2000, you
must reenter program mode. The two ways to do this are:

\begin{enumerate}
\item Type \textbf{BOOT} at the console prompt and press Retum. The System then
searches each device in turn for operating System Software.

\item Type \textbf{BOOT} followed by a space and the name of the device that
contains operating System Software and then press Return, as shown in
this example:

{\tt >>> BOOT DUA0}

This procedure lets the System boot the operating System Software immediately, without searching.

\end{enumerate}

See your Software documentation for more information.

\usubsection{Summary of TEST Commands}

A list of all TEST commands and the tests or Utility programs they execute
is shown in Table 3.

\begin{table}[H]
\caption{Summary of TEST Commands}
\label{table:3}
\begin{tabularx}{\textwidth}{l l}
\hline
\textbf{Command} & \textbf{Test or Utility Program}\\
\hline
TEST 0 & System exersizer \\
TEST F-1 & Self-tests of each device in the system \\
TEST 50 & Configuration \\
TEST 51 & Sets default boot device \\
TEST 52 & Sets default boot flags (operating system dependent) \\
TEST 53 & Sets default recovery action \\
TEST 54 & Sets keyboard language \\
TEST 70 & Diskette and fixed-disk formatter \\
TEST 71 & Fixed-disk verifier \\
TEST 81 & Screen of Es (color monitor) \\
TEST 82 & White screen (color monitor) \\
TEST 87 & IEight color bars (color monitor) \\
TEST 88 & Eight gray-scale bars (color or monochrome monitor) \\
\end{tabularx}
\end{table}


\howtoorder

\end{document}
