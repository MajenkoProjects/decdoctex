\chapter{FRU REMOVAL AND REPLACEMENT PROCEDURES}
\setcounter{page}{1}

\section{Introduction}
This chapter describes the removal and replacement procedures for the
field replaceable units (FRUs). To use this chapter, find the name of the
FRU that needs replacing in \tabref{3-1} then go to the section listed beside
the FRU. Follow the steps in the section to remove the FRU and reverse
the procedures to replace the FRU.

\begin{tbl}{FRU Section Listings}{l l l}
\textbf{FRU} & \textbf{Section} & \textbf{Page} \\
\hline

Battery pack								&	\npara{3.10}		&	3-33	\\[0.5em]
DEC423 Converter (MicroVAX 2000 only)		&	\npara{3.15.1}		&	3-53	\\[0.5em]
Expansion adapter							&	\npara{3.16}		&	3-54	\\[0.5em]
Expansion adapter disk interface module		&	\npara{3.16.1}		&	3-56	\\[0.5em]
Hard disk expansion box						&	\npara{3.17.1}		&	3-58	\\[0.5em]
Keyboard (VAXstation 2000 only)				&	\npara{3.13}		&	3-51	\\[0.5em]
Memory module								&	\npara{3.5}			&	3-17	\\[0.5em]
Mouse (VAXstation 2000 only)				&	\npara{3.14}		&	3-52	\\[0.5em]
Network interconnect module					&	\npara{3.6}			&	3-18	\\[0.5em]
Power supply in hard disk expansion box		&	\npara{3.17.1.2}	&	3-66	\\[0.5em]
Power supply in system box					&	\npara{3.11}		&	3-64	\\[0.5em]
Power supply in tape drive expansion box	&	\npara{3.17.2.3}	&	3-79	\\[0.5em]
RD32 disk drive								&	\npara{3.8}			&	3-24	\\[0.5em]
RD32 disk drive device electronics board	&	\npara{3.8.1}		&	3-27	\\[0.5em]

\end{tbl}

\begin{tblcont}{FRU Section Listings}{l l l}
\textbf{FRU} & \textbf{Section} & \textbf{Page} \\
\hline

RD53 disk drive										& \npara{3.7}					& 3-19	\\[0.5em]
RD53 disk drive device electronics board			& \npara{3.7.1}			& 3-21	\\[0.5em]
RD53 disk drive in hard disk expansion box			& \npara{3.17.1.1}	& 3-60	\\[0.5em]
Resistor load board in hard disk expansion box		& \npara{3.17.1.3}	& 3-70	\\[0.5em]
Resistor load board in system box					& \npara{3.9}					& 3-31	\\[0.5em]
Resistor load board in tape drive expnnsion box		& \npara{3.17.2.4}	& 3-81	\\[0.5em]
RX33 floppy disk drive								& \npara{3.8}					& 3-24	\\[0.5em]
System module										& \npara{3.4}					& 3-13	\\[0.5em]
Tablet (VAXstation 2000 only)						& \npara{3.14}				& 3-52	\\[0.5em]
Tape drive expansion box							& \npara{3.17.2}			& 3-72	\\[0.5em]
Terminal disconnection from MicroVAX 2000			& \npara{3.15}				& 3-53	\\[0.5em]
TK50 tape drive in tape drive expansion box			& \npara{3.17.2.2}	& 3-76	\\[0.5em]
TZK50 Controller board in tape drive expansion box	& \npara{3.17.2.1}	& 3-73	\\[0.5em]
VR260 deflection board								& \npara{3.12.2}			& 3-39	\\[0.5em]
VR260 monochrome monitor							& \npara{3.12}				& 3-35	\\[0.5em]
VR260 power LED										& \npara{3.12.7}			& 3-50	\\[0.5em]
VR260 rear bulkhead assembly						& \npara{3.12.3}			& 3-42	\\[0.5em]
VR260 transformer assembly							& \npara{3.12.5}			& 3-45	\\[0.5em]
VR260 video board									& \npara{3.12.4}			& 3-44	\\[0.5em]
VR260 tube/yoke/bezel assembly						& \npara{3.12.6}			& 3-47	\\[0.5em]
\end{tblcont}

\newpage

\figref{3-1} shows the locations of the FRUs as seen from the front.

\fig{MA-0133+0150-87}{FRU Locations}

\newpage

\section{Shield Removal}

\begin{enumerate}

\item	Turn the system power switch off and disconnect all cables from the
		rear of the system box.

\item	If the system is a MicroVAX 2000, remove the DEC423 converter from
		the rear of the system box.

\item	If the expansion adapter is installed, then remove the two screws shown
		in \figref{3-2} on the bottom of the expansion adapter to access the four
		screws securing the system box cover.

\fig{MA-0069-87}{Expansion Adapter Cover Screw Locations}
\newpage

\item	Remove the four system box cover screws shown in \figref{3-3}.

\fig{MA-0070-87}{System Box Cover Screw Locations}
\newpage

\item	Slide the cover off the system box as shown in \figref{3-4}.

\fig[0.6]{MA-0071-87}{System Box Cover Removal}
\newpage

\item	Put the system on its feet and remove the five screws shown in
		\figref{3-5} from the shield.

\fig{MA-0072-87}{Shield Screw Locations}
\newpage

\item	Lift the shield as shown in \figref{3-6} and rest it on the rear of the
		system.

\fig{MA-0073-87}{Removing the Shield from the Main Chassis}
\newpage

\item	Hold the shield with one hand and disconnect the power cable and the
		drive interconnect cable(s) from the shield with the other hand. Refer
		to \figref{3-7}.

\fig[0.9]{MA-0074-87}{Disconnecting the Cables from the Shield}

\item	Lift the shield off the main chassis.

\item	Return to the FRU removal instructions that sent you to this procedure.
\end{enumerate}

\newpage

\section{System Module Assembly Removal from Shield}

\begin{enumerate}

\item	Remove the two screws shown in \figref{3-8} from the shield.

\fig{MA-0075-87}{System Module Assembly Screw Locations on the Shield}
\newpage

\item	Disconnect the two network interconnect cable9 shown in \figref{3-9}
		(if installed).

\fig{MA-0076-87}{Network Interconnect Cables}
\newpage

\item	Remove the system module assembly from the shield as shown in 
		\figref{3-10} and lay it down beside the shield.

		\caution{Disconnection of the battery cable from the system module destroys
				 all NVR memory. Be careful not to disconnect the battery cable from the system
				 module until a procedure specifically instructs you to do so.}

\fig{MA-0077-87}{Removing System Module Assembly from Shield}
\end{enumerate}

\newpage

\section{System Module Removal}

\begin{itemize}

\item	Remove the shield (\para{3.2}).

\item	Remove the system module assembly from the shield (\para{3.3}).

\item	Disconnect the battery cable from the system module. Refer to \figref{3-11}.

\fig{MA-0078-87}{Battery Cable Location}
\newpage

\item	Remove the memory module (if installed). Refer to \figref{3-12}. Two
		connectors disconnect as you lift the memory module.
\fig{MA-0079-87}{Memory Module Removal}

\item	Remove the two network interconnect cables from the system module
		and install them on the new system module.

\newpage

\item	Remove the network ID ROM shown in \figref{3-13} from the system
		module you are removing and install it on the system module you are
		installing.
\fig{MA-0080-87}{Network ID ROM Location on System Module}
\newpage

\item	You must set the system jumper, shown in \figref{3-14}, for either a
		VAXstation 2000 or a MicroVAX 2000. Both systems use the same 
		system module and this system jumper teils the firmware whether the 
		module is in a VAXstation 2000 or a MicroVAX 2000. Determine which 
		system you are working on and set the jumper accordingly.
\fig{MA-0081-87}{VAXstation 2000 and MicroVAX 2000 System Jumper}

\item	Install the new system module by reversing the above procedures.

\end{itemize}

\newpage

\section{Memory Module Removal}

\begin{enumerate}

\item	Remove the shield (\para{3.2})

\item	Remove the system module assembly from the shield (\para{3.3})

\item	Remove the memory module shown in \figref{3-15}. Two connectors
		disconnect as you lift the module.

\fig{MA-0079-87}{Memory Module Removal}

\item	Replace the memory module by reversing the above procedures.

\end{enumerate}
\newpage

\section{Network Interconnect Module Removal}
\begin{enumerate}

\item	Remove the shield (\para{3.2})

\item 	Remove the system module assembly from the shield (\para{3.3})

\item	Remove the network interconnect module from the shield as shown in \figref{3-16}.

\fig{MA-0082-87}{Network Interconnect Module Removal}

\item	Replace the network interconnect module by reversing the above procedures.
\end{enumerate}

\newpage

\section{RD53 Hard Disk Drive Removal from System Box}

\begin{enumerate}

\item	Remove the shield (\para{3.2})

\item	Remove the three screws shown in \figref{3-17} that secure the drive to
		the chassis.

\fig{MA-0083-87}{Hard Disk Screw Locations}

\newpage

\item	Slide the drive out halfway and disconnect the drive interconnect cables
		and the power cable from the drive as shown in \figref{3-18}.

\fig{MA-0125-87}{Disconnecting the Hard Disk Drive from System Box}

\item	Slide the drive out of the chassis.

\newpage

\item	Remove two screws on each side of the drive mounting brackets shown
		in \figref{3-19}. Remove the mounting brackets and front panels from
		the drive.

\fig[0.8]{MA-0084-87}{Drive Mounting Bracket Screw Locations}

\item	If you removed the RD53 because of an error with the drive, go to
		\para{3.7.1}.

\item	Reverse the above procedures to replace the drive.
\end{enumerate}

\subsection{RD53 Disk Drive Device Electronics Board Removal}

The RD53 disk drive has an FRU inside it. It is the device electronics board.
Always replace the device electronics board on the RD53 before you replace
an entire RD53 disk drive.

\begin{enumerate}

\item	If a skid plate is installed on the drive, remove the four phillips screws
		securing the skid plate and ground clip to the frame. Remove the plate
		and set aside.

\item	Loosen the two captive screws that hold the device electronics board in
		place and then disconnect the ground wire.

\item	Rotate the board upward as shown in \figref{3-20} (the board pivots in
		hinged slots at the front of the drive). Be careful not to strain any of
		the connectors or cables, and tilt the board up and back until it rests
		against the outer frame.

		\caution{Flexible cirant material is fragile and requires careful handling to
				 avoid damage.}

\fig{SHA-0134-85}{RD53 Device Electronics Board Removal}

\item	Disconnect the two connectors from under the device electronics board.
		Both connectors and cables are fragile, handle them with care.

\item	Lift the board out of the hinged slots.

\newpage

\item	Position the jumpers and switches on the new board to the same 
Remove the shield (\para{3.2})

		position as the jumpers and switches on the old board you just removed.
		Refer to \figref{3-21}. Also be sure the new board has the resistor 
		terminator pack installed.

\fig{SHR-0126-85}{RD53 Device Electronics Board Jumper and Switch Locations}

\item	Install the replacement device eiectronics board by reversing the above procedure.

\end{enumerate}

\newpage

\section{RX33 and RD32 Drive Removal (Half-Height Drives) from the System Box}

\begin{enumerate}

\item	Remove the shield (\para{3.2})

\item	Remove the three screws shown in \figref{3-22} that hold the drive(s) to
		the chassis.

\fig[0.8]{MA-0085-87}{Half-Height Drive(s) Screw Locations}
\newpage

\item	Slide the drive(s) out halfway and disconnect the drive interconnect
		cable(s) and the power cable(s) from the drive(s). Refer to \figref{3-23}.

\fig{MA-0086-87}{Disconnecting the RX33 and RD32 Half-Height Drives}

\item	Slide the drive(s) out of the chassis.
\newpage

\item	Remove four screws on each side of the mounting brackets shown in
		\figref{3-24}. Remove the mounting brackets and front panels from the
		drive(s).

\fig{MA-0087-87}{Half-Height Drives Mounting Brackets Removal}

\item	If you are replacing the RX33 floppy drive, install the new RX33 by
		reversing the above procedures.

\item	If you removed the RD32 because of an error with the drive, go to
		\para{3.8.1} below. Then reverse the 
		above procedure to reinstall the drive.

\end{enumerate}
\newpage

\subsection{RD32 Disk Drive Device Electronics Board Removal}

The RD32 disk drive has an FRU inside it. It is the device electronics board.
Always replace the device electronics board on the RD32 before you replace
an entire RD32 disk drive.

\caution{Do not remove any of the screws that secure the cover (cover/filter
assembly) to the base casting. Removing any of the screws violates the clean area.}

\begin{enumerate}
\item	Remove the three flat head screws shown in \figref{3-25} that hold the
		device electronics board in place under the drive unit. Besure to locate
		the one screw with the nylon washer next to the drive motor when
		replaceing the board.

\fig{SHR-0296-86}{RD32 Device Electronics Board Screw Locations}
\newpage
\item	Disconnect the cable from J4 (middle of board) as shown in \figref{3-26}.
		The J4 connector is not keyed and can be reconnected incorrectly. Make
		sure the violet and yellow wire in the connector is located near the LED
		and the orange and blue wires are located toward the outer edge of the
		board when reconnecting J4.
\fig{SHA-0297-86}{Disconnecting J4 from RD32 Device Electronics Board}

\item	Stand the board up. Be careful not to strain any of the connectors or
		cables, and tilt the board over center until it rests against the outer
		frame.

		\caution{Flexible circuit material is fragile and requires careful handling to
				 avoid damage.}
\newpage
\item	Disconnect the cables from J6 and J5 as shown in \figref{3-27} from
		the device electronics board. Both connectors and cables are fragile --
		handle them with care.

\fig{SHR-0298-86}{Disconnecting J5 and J6 from RD32 Device Electronics Board}

\item	Remove the device electronics board from the RD32.
\newpage

\item	Position the jumper on the new board to the same position as the jumper
		on the old board you just removed (\figref{3-28}). Also be sure the new
		board has the resistor termination pack installed.

\fig{MA-0131-87}{RD32 Device Electronics Board Jumper Location}

\item	Replace the new device electronics board by reversing the above procedures.
\end{enumerate}

\section{Resistor Load Board Removal from System Box}

The resistor load board loads down the power supply in a diskless system
or when just the RX33 floppy drive is installed in the system box. The
power supply does not have enough of a load on it to regulate properly
in these configurations. The load board simulates the drives by drawing
enough power to regulate the power supply. You must remove the load
board whenever a hard disk drive is installed in the system box.

The resistor load board in the system box is different from the resistor load
board in the expansion boxes. The part number of the resistor load board in
the system box must be 5417163-02. The -01 version goes in the expansion
boxes.

\begin{enumerate}

\item	Remove the shield (\para{3.2}).

\item	Remove the three screws shown in \figref{3-29} that secure the drive
		mounting bracket to the chassis.

\item	Disconnect the power cable from the resistor load board.

\item	Lift the load board up and out of the chassis.

\item	If you are removing the load board to install a drive, remove and 
		discard the standoffs that the load board sat on, then install the drive by
		reversing the removal instructions for that drive. Otherwise, install the
		new load board (p/n 5417163-02) by reversing the above procedures.

\end{enumerate}

\newpage

\fig{MA-0118-87}{Drive Mounting Bracket Screw Locations}

\newpage

\section{Battery Pack Removal}

\begin{enumerate}

\item	Remove the shield (\para{3.2})

\item	Remove the system module assembly from the shield (\para{3.3})

\item	Disconnect the battery cable from the system module.

\fig{MA-0078-87}{Battery Cable Location}

\item	Take the battery pack out of its holder.

\item	Replace the battery pack by reversing the above procedures.

\note{A new battery needs a minimttm of 25 hours of continous power on to
fully charge the battery. If the battery's charge is low, you will see an error for
the NVR when you power up the system.}

\end{enumerate}

\newpage

\section{Power Supply Removal from the System Box}

\begin{enumerate}
\item	Remove the shield (\para{3.2}).

\item	Disconnect the power cables from the rear of the drives. If you need
		more room behind the drives, remove the three drive mounting screws
		and slide the drives out part way.

\item	Put the system on its side and remove the four screws shown in
		\figref{3-31} that hold the power supply to the bottom of the chassis.

\fig{MA-0127-87}{Power Supply Screw Locations in the System Box}
\newpage
\item	Slide the power supply out of the chassis.

\item	Replace the power supply by reversing the above procedures.

\end{enumerate}

\section{VAXstation 2000 VR260 Monochrome Monitor}

You may need the following tools to repair the VR260 Monitor.

\begin{itemize}
\item	Phillips screwdriver
\item	Slotted screwdriver
\item	High-voltage discharge tool (p/n 29-24717)
\end{itemize}

\figref{3-32} shows the FRUs in the VR260 monitor.

\fig{MA-0088-87}{VR260 FRU Locations}
\newpage

\subsection{VR260 Cover}

\begin{enumerate}

\item	Switch the monitor power switch off.

\item	Disconnect the power cord and video cable from the monitor. Refer to \figref{3-33}.

\fig{MA-7557-80}{VR260 Controls and Video Cable Locations}

\item	Carefully place the monitor on its face.

\item	Loosen the four rubber feet on the bottom.

\item	Slide the tilt-swivel base up and off the monitor.

\item	Finish unscrewing the four rubber feet and remove them from the monitor.

\newpage

\item	Remove the four phillips cover screws shown in \figref{3-34} from the rear of the monitor.

\fig{MA-0089-87}{VR260 Cover Screw Locations}

\item	Lift the cover off.

\newpage

\item	Lift the bezel off. Refer to \figref{3-35}.

\fig{MA-0090-87}{VR260 Rear Bezel Location}

\item	Place the monitor back down to its normal position.

\item	Reverse the above procedures to replace the cover.

\end{enumerate}
