\chapter{Testing and Troubleshooting}
\setcounter{page}{1}

\section{Introduction}

This chapter describes how to test and troubleshoot the VAXstation 2000
and the MicroVAX 2000 systems. Differences between the VAXstation 2000
and the MicroVAX 2000 are specifically identified in the text.

This chapter contains the following sections.

\begin{itemize}
\item How to use console mode: determining the console device, enter
ing console mode, exiting console mode, utilizing the diagnostic
console device, and where to find a list of the console commands.
\item How to run the diagnostic tests and interpret the error codes for
each test: power-up tests, self-test, and system exerciser.
\item How to troubleshoot all devices in the system.
\item How to use the utilities.
\end{itemize}

\section{Using Console Mode}

The VAXstation 2000 and the MicroVAX 2000 systems have two modes
of operation: program mode and console mode. Normal operation of the
VAXstation 2000 and the MicroVAX 2000 is in program mode, that is, with
the operating system controlling the system. Console mode allows the user
to control the system from the console terminal using the console com
mands described in \hyperlink{appendix.b}{Appendix B}. Console mode is contained in ROM on
the system module.

Testing is done while in console mode. The System returns the \console
prompt when it is in console mode. \hyperref[table:2-1]{Table 2-1} lists the prompts and the
mode of operation each prompt represents.

\begin{table}[H]
\caption{Prompts}
\label{table:2-1}
\small
\begin{tabularx}{\textwidth}{c l}
\hline
\textbf{Prompt} & \textbf{Mode of operation}\\
\hline
{>}{>}{>} & Console mode. Console commands are listed in \hyperlink{appendix.b}{Appendix B}. \\
\$ & Program mode (VMS operating system) \\
\% & Program mode (Ultrix operating system) \\
\hline
\end{tabularx}
\end{table}

\subsection{Determining the Console Device}

The console device for a VAXstation 2000 system is the keyboard (LK201)
and monitor (VR260) connected to the video port. The keyboard inputs
commands at 4800 baud and the monitor displays output from the video
circuits.

The console device for a MicroVAX 2000 system is the terminal (VT220 or
similar terminal) connected to connector 1 on the DEC423 converter. The
terminal operates at 9600 baud.

\subsection{Entering Console Mode}

Console mode is entered any time the CPU halts. The CPU can be halted
automatically or manually. A halt means that CPU control has passed 
control from the operating system to the console mode program in ROM. If
the system halts the CPU, then the console mode program checks the 
nonvolatile RAM (NVR) for user-defined instructions on how to handle the halt.
If you manually halt the CPU, the system enters console mode program 
immediately without checking the NVR for instructions.

You can manually halt the CPU and enter console mode by one of the
following methods.

\begin{itemize}
\item HALT button -- Press the halt button. It is located next to the printer port
on the back of the System box. The \console prompt is displayed when
ready for console commands.
\item BREAK key -- Press the BREAK key on the diagnostic console device
that is connected to the printer port with the BCC08 cable. The \console
prompt is displayed when ready for console commands.
\end{itemize}
The system automatically halts the CPU for the following reasons.
\begin{itemize}
\item After power-up testing -- If the default recovery action is halt, the system
automatically halts the CPU and enters console mode after power-up
tests are complete. See Section 2.5.4 for information on setting the
default recovery actions.
\item For a boot failure -- If the system fails to boot properly, the system 
automatically halts the CPU and enters console mode. See Section 2.5.2
for information on setting the default boot device.
\item On a system error -- If the CPU detects a severe corruption of its oper-
ating environment, it halts and reads the default recovery action in the
NVR. The default recovery action can be restart, boot, or halt. When
it is restart and the restart fails, then the system automatically tries to
boot the operating system software. If the boot fails, the CPU halts and
enters console mode. When the default recovery action is boot and the
boot fails, the CPU halts and enters console mode. When the default re
covery action is halt, the CPU unconditionally halts and enters console
mode.
\end{itemize}

One other way to halt the CPU is when the operating system software 
executes a halt instruction. The CPU then reads the default recovery action in
the NVR and acts on it as described above.

\subsection{Exiting Console Mode}

Console mode is exited by typing one of the following console commands.

\begin{itemize}
\item BOOT -- This command initializes the CPU and boots the operating 
system software from the device specified. If no device is specified, the
system searches each on-line device until the operating system software
|s found. The boot command starts the system similar to when power
is turned on except that the power-up tests are not run. If the system
attempts to boot over the net (ESA0) and no software is available on
another node, the system keeps looking for the software indefinitely.
If a boot message for the operating system software does not appear
shortly after the Ethernet boot message (ESA0) is displayed, then you
must press the halt button to abort the Ethernet boot. If you still need
to boot over the Ethernet, make sure the node with the operating sys-
tem software is operating normally and the software is loaded. Run the
Ethernet loopback Utility (TEST 90) to check the networking capability
of the system if the Ethernet boot continues to fail.

When a boot is invoked using this boot command, you can specify
several boot command flags by bit encoding the flags in a flag word
specified with the /R5: qualifier. These command flags are listed in
Paragraph 2.5.3.

\item CONTINUE -- This command instructs the CPU to continue the 
operating system software at the address contained in the program counter
(PC). This command starts up the operating software where it was halted
provided no console test commands were run. Running a test command
alters the PC and memory so that the operating system software cannot
be started properly by entering the continue command. If a test com-
mand was entered, use the boot command to start the operating system
software. The continue command is similar to the start command. The
start command can specify the address to Start the operating software
and the continue command has the operating software continue where
it left off.

\item START -- This command Starts the operating System Software at a 
specified address. If no address is given, the contents of the PC are used.
However, running a test command alters the PC and memory so that the
operating system software cannot be started properly by entering the
start command. If a test command was entered, use the boot command
to start the operating system software.

\end{itemize}

\subsection{Diagnostic Console Device}

There is a diagnostic console device available on the VAXstation 2000. The
MicroVAX 2000 can use this device if the DEC423 Converter is removed.

The diagnostic console device can be utilized by connecting a terminal (such
as the VT100 or VT220) to the printer port with the special BCC08 cable.
The terminal operates at 9600 baud. Field service technicians can use this
terminal as a diagnostic tool to isolate a problem in the normal console
device.

To use this diagnostic console device you must turn off power, connect the
BCC08 cable to the printer port and terminal, and then turn power back on.
The diagnostic console device now controls the system. The normal console
monitor displays video test patterns on the VAXstation 2000 when the video
circuits are tested. The normal console keyboard does not operate.

\subsection{Console Commands}

Console commands are listed in \hyperlink{appendix.b}{Appendix B}.
\newpage
\section{Testing}

Testing procedures on the VAXstation 2000 and the MicroVAX 2000 systems
are almost identical. The differences between the systems are explained
where applicable. All diagnostic tests are ROM-based and testine is done
while in console mode. Tests are executed in either of two ways:

\begin{itemize}
\item Automatically -- When power is turned on the power-up tests begin.
\item Manually -- By entering one of the console test commands on the 
console terminal. See \hyperlink{appendix.a}{Appendix A} for a complete listing of the test commands.
\end{itemize}

\subsection{Power-up Tests}

Power-up tests run each time the system power is turned on. Power-up
testing consists of a sequence of tests executed for each device installed in
the system. The test number of each device is listed on the power-up screen
display as the device is tested. Figure 2-1 shows an example of the power-
up screen display. The first line indicates the CPU name (KA410-A) and the
ROM version (V1.0). The test numbers are listed next in descending order
from the first test, F, to the final test, 1. TEST F will have an underscore
after it on the MicroVAX 2000 to indicate that TEST F was not run. Note in
Figure 2-1 that tests 4, 3, 2, and 1 have an underscore (\_) immediately after
them. This underscore indicates that there is no option device installed for
that test; thus, no tests are done. TEST F has an underscore after it on the
MicroVAX 2000 systems because the monochrome video circuits are not
used by the MicroVAX 2000. No other test numbers can have underscores
after them. An asterisk (*) after TESTS 4 through 1 indicate that an option is
installed, but its ROM is destroyed and the Option device must be replaced.
Only TESTS 1 through 4 can have an asterisk after them. Figure 2-2 lists
the symbols that can appear in between the tests and what they indicate.

\begin{ttfig}{2-1}{Example of Power-up Tests Screen Display}
KA410-A V1.0
F...E...D...C...B...A...9...8...7...6...5...4_..3_..2_..1_..
\end{ttfig}

Figure 2-2 lists the definitions of the symbols that appear between the test
numbers in the power-up test countdown.

\begin{ttfig}{2-2}{Power-up Symbols Defined}
    ...  Device tested successfully or has a soft error
    ?..  Device has a hard error
    _..  Device not installed or not tested
    *..  Device installed but its ROM is destroyed
\end{ttfig}

If any hard errors (errors that indicate the device must be replaced for
proper operation) are found during power-up testing, a question mark is
placed after the failing test number during the countdown sequence. An
error summary of all errors detected is listed after the power-up sequence
is complete. Two question marks in the error summary indicate a hard error.
Error codes that indicate the status or soft errors do not put a question mark
after the failing test number in the sequence, but do list the error code in the
error summary. Figure 2-3 shows the power-up screen display with a hard
error found in TEST F and a soft error found in TEST E. The error summary
for each failed device is displayed before the boot sequence is started.
However, the screen usually scrolls so fast when the system starts to boot
that you may not be able to see what the error summary contained (if there
was an error summary). To see what errors the power-up tests found, press
the halt button and enter TEST 50 on the console terminal. TEST 50 is the
command for bringing up the configuration table. The configuration table
is created during power-up testing. This configuration table contains all of
the error codes listed in the power-up error summary as well as error codes
for all devices installed in the system. The error codes in the configuration
table are updated every time self-test is run. See Paragraph 2.5.1 for an
explanation of how to use the configuration table.

Each error summary consists of one or two question marks, a test number,
the ID number of the failed device, and an eight-digit error code. For ex-
ample, in \hyperref[figure:2-3]{Figure 2-3}, the first line of the error summary shows a hard error
for TEST F, a device ID number of 00B0, and an error code of 0001.F002.
The second line shows a soft error for TEST E, a device ID of 0040, and an
error code of 0000.0005. \hyperlink{subsubsection.2.3.1.1}{Section 2.3.1.1} describes the error codes.
\newpage
\begin{ttfig}{2-3}{Example of Power-up Tests Screen Display with Errors.}
KA410-A V1.0

F?..E...0...C...B...A...9...8...7...6...5...4_..3_..2_..1_..

?? F  00B0  0001.F002
 ? E  0040  0000.0006
\end{ttfig}

If there is a fatal error in the NVR during power-up testing, the system stops
testing the other devices and displays ?14 TOY ERR on the screen. When
this happens, the only way to determine the cause of the problem is by
viewing the LEDs on the keyboard. One of the LEDs will be lit to indicate
the failing module. \hyperref[table:2-2]{Table 2-2} lists the LEDs and which module has failed.

\begin{table}
\label{table:2-2}
\caption{Keyboard LEDs Deflned}
\small
\begin{tabularx}{\textwidth}{l l}
\hline
\textbf{Keyboard LED} & \textbf{Failing module}\\
\hline
Hold Screen & System module \\
Lock & Not used \\
Compose & ThinWire Ethernet option module \\
Wait & Not used \\
\hline
\end{tabularx}
\end{table}

Once power-up testing is complete and no fatal or hard errors are found, the
system boots the operating system software. Both the VAXstation 2000 and
the MicroVAX 2000 boot the operating system software the same. The only
difference between the two is that the VAXstation 2000 clears the console
screen before it boots and the MicroVAX 2000 does not. If a default boot
device is loaded in the NVR, the system boots off of that device. If no
default device is loaded in the NVR, the system searches every on-line
storage drive for the operating software. DUA2 is searched first if a floppy
diskette is loaded. Otherwise, it is not searched at all. The hard disks are
searched next, DUA0 then DUAl. MUA0, the tape drive, is checked after
the hard disk drives if it is installed and a cartridge is loaded. Finally, the
system searches the Ethernet network for the software and ESA0 is listed
on the screen. The system continues to search the Ethernet network until
the operating system software is found.
\newpage
\subsubsection{Power-up Test Error Codes}

The power-up test error codes indicate status and/or error information. Any
errors found by power-up tests are listed in the error summary after the
power-up test countdown sequence. This summary, if any, gives you a brief
summary of the errors. \hyperref[table:2-3]{Table 2-3} lists the test numbers and the devices that
are tested during that particular test. To look at the complete list of devices
and the status of that device, you must display the configuration table. The
configuration table lists every device in the system and also lists the results
of the self-test and power-up tests and is updated each time self-test is run.
The error codes for each device in the configuration table are explained in
the troubleshooting section for that individual device. Remember that the
configuration table contains the results of the self-test and power-up tests
and not the results of the system exerciser. \hyperref[figure:2-4]{Figure 2-4} shows an example
of the configuration table and for an explanation of the configuration table,
see \hyperlink{subsection.2.5.1}{Paragraph 2.5.1}.


\begin{ttfig}{2-4}{Example of the Configuration Table}
>>> TEST 50

KA410-A V1.0
ID 08-00-2B-02-CF-A4

?? MONO       0001.F002
 ? CLK        0000.0005
   NVR        0000.0001
   DZ         0000.0001
       00000001 00000001 00000001 00000001 00000001 000012A0
   MEM        0002.0001
       00200000
   MM         0000.0001
   FP         0000.0001
   IT         0000.0001
   HDC        1710.0001
       000146B8 00000000 00000320
   TPC        0202.0001
       FFFFFF03 01000001 FFFFFF06 FFFFFF05 FFFFFF05 FFFFFF05 ...
   SYS        0000.0001
   NI         0000.0001 V1.0
>>>
\end{ttfig}
\newpage
The most common good error code is 0000.0001. There are, however, some
devices that use the first four digits in the error code to indicate the status
of the device and the last four digits to indicate the error found on the
device. The memory (MEM) error code, for instance, contains 0002.0001
which indicates two megabytes of memory is available (0002.) and no error
found (.0001). On devices like these, the last four digits always indicate
.0001 as a good (non-error) indication.

Some error codes indicate no error at all and give a status of the device
such as the clock (CLK) which shows that the date and time has not been
set. This is not an error, just a status of the clock circuits.

Any error code other than 0000.0001 on the MONO, MM, FP, IT, or SYS
devices indicates a hard error and that device must be replaced for proper
operation of the system. The other devices such as CLK, NVR, DZ, MEM,
HDC, TPC, and NI may have a status or a soft error message in the error
codes and may still operate normally.

See the troubleshooting procedures section (Paragraph 2.4) for each device
to determine whether or not the error code indicates a fault or a status for
the device.

\subsection{Self-test}

Self-test allows you to test every device again individually, a few at a time,
or all of them sequentially just like power-up tests. To individually test a
device, enter TEST \# where \# is the test number of the device you want
tested. \hyperref[table:2-3]{Table 2-3} lists the test numbers and the devices tested by those
numbers. Figure 2-5 shows an example of running self-test successfully on
the disk controller.

\begin{ttfig}{2-5}{Example of Running Self-test on the Disk Controller}
>>> TEST 7
  7...
>>>
\end{ttfig}

\newpage

To test a group of devices, enter TEST followed by the test number of the
first device to be tested and then the test number of the last device to be
tested. Figure 2-6 shows an example of testing a group of devices. In Figure
2-6, all tests between C and 4 are tested successfully. Note that you cannot
pick and choose which devices to test between C and 4, all tests between
C and 4 are tested when entered as a group.

\begin{ttfig}{2-6}{Example of Running a Series of Self-tests}
>>> TEST C 4
  C...B...A...9...8...7...6...5...4_..
>>>
\end{ttfig}

To test all devices, enter TEST F 1. The MicroVAX 2000 skips over the
MONO video test (TEST F) since it does not use the video circuits.

\begin{table}[H]
\caption{Self-test Commands}
\label{table:2-3}
\small
\begin{tabularx}{\textwidth}{p{2cm} p{8cm}}
\hline
\raggedright\textbf{Test\newline Number} & \textbf{Device\newline Tested} \\
\hline
1 & Option module (Network Interconnect module) (NI) \\[0.5em]
2 & Option module (not available) \\[0.5em]
3 & Option module (not available) \\[0.5em]
4 & Option module (not available) \\[0.5em]
5 & Interrupt Controller and ThinWire Ethernet ID ROM (SYS) \\[0.5em]
6 & Tape Controller. (TPC) \\[0.5em]
7 & Disk conroller. (HDC) \\[0.5em]
8 & Interval timer. (IT) \\[0.5em]
9 & Floating point unit. (FP) \\[0.5em]
A & Memory management unit. (MM) \\[0.5em]
B & Memory. (MEM) \\[0.5em]
C & DZ Controller. (DZ) \\[0.5em]
D & Non-volatile RAM. (NVR) \\[0.5em]
E & Time-of-year dock. (CLK) \\[0.5em]
F & Base video (MONO) (VAXstation 2000 only) \\[0.5em]
\hline
\end{tabularx}
\end{table}

\subsubsection{Self-test with Loopback Connectors}

Customer mode self-test does not test the drivers or the lines of the serial
line conroller (DZ) since loopbacks are not used. Run self-test in field 
service mode to test the DZ drivers by installing the loopback connectors on
the back of the System. Follow one of the procedures below.

To test the DZ on VAXstation 2000, install a loopback (p/n 29-24795) on
the 25-pin communication port and a loopback (p/n 29-24794) on the 9-pin
printer port. Run TEST C. You cannot use loopback connectors if you are
using the diagnostic console device with the BCC08 cable on the printer
port since there is no loopback connector for the video port.

To test the DZ on MicroVAX 2000, install a loopback (p/n 29-24795) on the
25-pin communication port and install an MMJ loopback on both ports 2
and 3. Run TEST C.

\textbf{NOTE:} \textit{The ThinWire Ethernet port on the back of the 
system box must be terminated properly when running diagnostics on the 
network option (TEST 1) othetwise an error code of 0000.7001 or greater 
is listed in the configuration table.}

\subsubsection{Self-test Error Codes}

Figure 2-7 shows how an error is displayed if found during self-test. This
example shows an error on the disk controller during self-test. The 84 FAIL
indicates an error was found on the device tested. You must display the
configuration table (TEST 50) after self-test is complete to see the error code,
if there is an error during self-test, since the error codes do not appear on
the screen. The configuration table lists every device in the system, fists the
results of the self-test and power-up tests, and is updated each time self-test
is run. The error codes for each device in the configuration table are 
explained in the troubleshooting section for that individual device. Remember
that the configuration table contains the results of the self-test and power-up
tests and not the results of the system exerciser. See Paragraph 2 5 1 for an
explanation of the configuration table.

\newpage 
\begin{ttfig}{2-7}{Example of a Self-test Error on the Disk Controller}
  >>> TEST 7
    7?..
    84 FAIL
  >>>
(You must display the configuration table to see the error code)
\end{ttfig}

\newpage

\subsection{System Exerciser Diagnostics}

The system exerciser simulates a worst-case situation test for each device
and checks how the device operates under these conditions. It does not use
the configuration table to list its results of the tests. The system exerciser
has a separate display that appears on the screen as the exerciser is running.
Any errors found are displayed in the exerciser display. When examining
the exerciser display, a single question mark in the far left column indicates
a soft error, a double question mark indicates a hard error, and the absence
of question marks indicate success. \hyperref[figure:2-8]{Figure 2-8} shows an example of the
exerciser display.

The system exerciser exercises most of the devices. However, some devices
such as the memory management unit (MM) and the interval timer (IT) are
tested through the testing of other devices and are not displayed. Of the
devices the exerciser does exercise, it runs each one sequentially until all
have been run once, then it runs them all at the same time (worst-case).
This type of testing usually finds any intermittent failures.

The system exerciser has two modes: customer mode and field service
mode. Customer mode system exerciser (TEST 0) does not use loopback
connectors and does not fully test all of the devices. Field service mode sys-
tem exerciser requires loopback connectors installed and removable media
from the maintenance kit inserted and loaded. You must initialize the floppy
diskette in the maintenance kit with a special diagnostic key so the exerciser
can perform write tests on the RX33. Refer to Paragraph 2.5.11 for more
Information on creating the special diagnostic keys. Once the floppy has
been initialized, load it into the RX33 (if a full read/write test of the RX33 is
necessary) before you run the system exerciser in field service mode. You
must also initialize the COMPACTape cartridge in the maintenance kit with
a special diagnostic key so the exerciser can perform read/write tests on
the TK50. Refer to Paragraph 2.5.11 for more information on creating the
special diagnostic keys. Once the COMPACTape cartridge has been initial
ized, load it into the TK50 (if a full read/write test of the TK50 is necessary)
before you run the system exerciser in field service mode. If the RX33 or
the TK50 are not loaded with the special-key media, the system exerciser
does not do destructive writes to them and tests them the same as it does
during the customer mode system exerciser. This special-key on the media
prevents the exerciser from accidentally destroying data on the customers
floppy diskette or COMPACTape cartridge. The field service mode system
exerciser is available in a run once (TEST 101) and a run forever (TEST 102)
configuration.

\newpage

\subsubsection{System Exerciser Diagnostic Commands}
\hyperref[table:2-4]{Table 2-4} lists the System exerciser diagnostic commands.

\begin{table}[H]
\label{table:2-4}
\caption{System Exerciser Diagnostic Commands}
\small
\begin{tabularx}{\textwidth}{p{0.2\textwidth} p{0.8\textwidth}}
\hline
\textbf{Test Commands} 	&	\textbf{Description of Commands} \\
\hline

0						&	Runs customer mode System exerciser. It exercises 
							each device once sequentially, then exercises them 
							simultaneously, and stops when the slowest device
							finishes its second pass. No loopback connectors 
							and no removable media required. \\

101						&	Runs field service system exerciser. It exercises 
							each device once sequentially, then exercises them 
							simultaneously, and stops when the slowest device 
							finishes its second pass. Do not stop the exerciser 
							before it finishes exercising every device twice 
							(second pass). Loopbacks and removable media required. \\

102						&	Runs field Service System exerciser. It exercises each 
							device once sequentially and then exexcises them 
							simultaneously until you enter a \keystroke{CTRL}-\keystroke{C}. 
							Note that the exerciser takes up to thirty seconds 
							to stop after you enter \keystroke{CTRL}-\keystroke{C}. Do
							not stop the exerciser until every device is exercised 
							twice (second pass). Also, do not press the halt button 
							to stop the exerciser. Loopbacks and removable media required. \\
\hline
\end{tabularx}
\end{table}

When the exerciser is started, PRA0 is displayed and the monitor connected
to the video port on the VAXstation 2000 blinks white and black several
times while the monochrome circuits are being tested. The results of the
MONO tests are then displayed on the console screen. MicroVAX 2000
does not use the MONO circuits and, therefore, does not test them. The
DZ test results are the first to be displayed on the console screen. The rest
of the devices are then exercised one at a Urne and the results are listed
on the display. The console displays the results of each device until the
last device is finished testing. When the last device is done, the exerciser
starts running all devices together at the same time. When this happens, the
monitor connected to the video port starts blinking again until the slowest
device finishes testing. On the MicroVAX 2000, the console terminal holds
the first pass display until the slowest device is done testing then displays a
new exerciser display. The console screen (VAXstation 2000 and MicroVAX
2000) displays the results of each device tested. This display stays on the
screen for about 10 seconds, then the exerciser starts running all devices
together again if TEST 102 was entered. If TEST 0 or TEST 101 was entered,
the exerciser stops after the slowest device is done testing. The halt message
is displayed when the exerciser is stopped.

To run the field service mode system exerciser on VAXstation 2000, install a
loopback (p/n 29-24795) on the 25-pin communication port and a loopback
(p/n 29-24794) on the 9-pin printer port. Insert and load the floppy diskette
and TK50 COMPACTape cartridge from the maintenance kit if the system
has these devices installed. Run the exerciser by entering TEST 101 or TEST
102. If you are using the diagnostic console device, you will get errors on
the DZ line since there is no loopback for the video port.

To run the field service mode system exerciser on MicroVAX 2000, install
a loopback (p/n 29-24795) on the 25-pin communication port and install an
MMJ loopback on both ports 2 and 3. Insert and load the scratch floppy
diskette and scratch TK50 COMPACTape cartridge from the maintenance
kit if the system has these devices installed. Run the exerciser by enter
ing TEST 101 or TEST 102. If the scratch floppy or TK50 cartridge is not
installed, the system tests those devices the same as in customer mode.

\subsubsection{System Exersiser Error Codes}

\hyperref[figure:2-8]{Figure 2-8} shows an example of the system exerciser display while running
the system exerciser in field service mode. Customer mode gives the same
display, but with a CU in place of the FS on the top of the display.

\begin{ttfig}{2-8}{System Exerciser Display Example}
KA410-A V1.0          01     FS

   F  00B0    MONO    0000.0001      2      0 00:00:55.01
   C  0080    DZ      0000.0001      5      0 00:01:28.03
   B  0010    MEM     0175.0001      2      0 00:02:03.07
   7  0090    HDC     2000.0001      4      0 00:02:08.58
?                     0700.7091             0 00:02:08.58
??                    1002.0051             0 00:01:03:45
                      2500.0001
   6  00A0    TPC     1000.0001      9      0 00:02:44.04
                      1300.0001
?? 1  OOCO    NI      0000.7004      3      0 00:04:46.32

>>>

  (The error codes are defined in the troubleshooting
   procedures for each individual device)
\end{ttfig}

The first line indicates the CPU name (KA410-A), the ROM version (V1.0),
the ROM Status (01 -- the ROM is corrupted if this is anything other than
01), and the mode of the system exerciser (CU for customer and FS for field
service mode). The next several lines list information on the devices that
are exercised by the system exerciser.

The first column of the display lists the test number (F,C,B,7,6,1). TESTS 4
through 1 are option modules and these options may not be installed. They
are listed in the display only if they are installed. The second column lists
the device identifier (00B0, 0080, 0010,...). Next is the mnemonic for each
device (MONO, DZ, MEM,...) followed by the error code (0000.0000) for
that device (the HDC and TPC devices have additional error codes for each
drive). After the error code is the number of times the device was tested
followed by the time of the last pass the exerciser made on that device.
The time is in the format of days hours:minutes:seconds.hundredths of
seconds. Question marks identify hard (??) and soft (?) errors. See the
troubleshooting procedures (\hyperlink{section.2.4}{Paragraph 2.4}) for 
the device that has an error detected with it.

\section{Troubleshooting Procedures}

Each section below contains troubleshooting information for the device 
indicated. Find the section below with the mnemonic (such as MONO or TPC)
that you want to troubleshoot. Once the section is located, read through the
procedures to familiarize yourself with the testing of the device.

\subsection{MONO -- Monochrome Video Troubleshooting Procedures (VAXstation 2000 only)}

You can troubleshoot the monochrome circuitry on the system module
in either customer mode or field service mode. Both modes test the
monochrome circuits the same. This procedure does not troubleshoot the
video monitor, just the video circuits on the system module.

\subsubsection{Self-test}

To run self-test on MONO, enter TEST F. The monitor connected to the
video port blinks white and black several times. This is normal. The results
of the self-test are displayed when self-test is complete. Any error code
other than 0000.0001 indicates a fault in the monochrome video circuits.
You must replace the system module to fix this problem.

\subsubsection{System Exerciser}

To run the system exerciser, enter TEST 0 for customer mode or TEST 101
for field service mode. The monitor connected to the video port blinks white
and black several times while the monochrome circuits are being tested. The
results of the MONO tests are displayed on the console screen after they are
tested for the first time, then displayed again after each concurrent testing.
Any error code other than 0000.0001 indicates a fault in monochrome video
circuits. You must replace the system module to fix this problem. This
device is not exercised on the MicroVAX 2000 system since MicroVAX 2000
does not use video monitors.

\subsection{CLK -- Time-of-Year Clock Troubleshooting Procedures}

You can troubleshoot the CLK circuitry in either customer mode or field
service mode. Both modes test the CLK circuits the same.

\subsubsection{Self-test}

To run self-test, enter TEST E. Any error code other than 0000.0001 
indicates a fault in the CLK circuits. An error code of 0000.0005 indicates that
the system time is not set. Refer to the operating system software 
documentation to set the time. If any other error code appears in the CLK error
display, you must replace the system module.

The system exerciser does not display the Status of the CLK circuits. The
CLK circuits are not directly tested but are tested through the testing of
other circuits.
