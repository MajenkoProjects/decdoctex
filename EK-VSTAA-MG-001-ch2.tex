\chapter{Testing and Troubleshooting}
\setcounter{page}{1}

\section{Introduction}

This chapter describes how to test and troubleshoot the VAXstation 2000
and the MicroVAX 2000 systems. Differences between the VAXstation 2000
and the MicroVAX 2000 are specifically identified in the text.

This chapter contains the following sections.

\begin{itemize}
\item How to use console mode: determining the console device, enter
ing console mode, exiting console mode, utilizing the diagnostic
console device, and where to find a list of the console commands.
\item How to run the diagnostic tests and interpret the error codes for
each test: power-up tests, self-test, and system exerciser.
\item How to troubleshoot all devices in the system.
\item How to use the utilities.
\end{itemize}

\section{Using Console Mode}

The VAXstation 2000 and the MicroVAX 2000 systems have two modes
of operation: program mode and console mode. Normal operation of the
VAXstation 2000 and the MicroVAX 2000 is in program mode, that is, with
the operating system controlling the system. Console mode allows the user
to control the system from the console terminal using the console com
mands described in \hyperlink{appendix.b}{Appendix B}. Console mode is contained in ROM on
the system module.

Testing is done while in console mode. The system returns the \console
prompt when it is in console mode. \tabref{2-1} lists the prompts and the
mode of operation each prompt represents.

\newpage

\begin{tbl}{2-1}{Prompts}{c l}
\textbf{Prompt} & \textbf{Mode of operation}\\
\hline
{>}{>}{>} & Console mode. Console commands are listed in \hyperlink{appendix.b}{Appendix B}. \\
\$ & Program mode (VMS operating system) \\
\% & Program mode (Ultrix operating system) \\
\end{tbl}

\subsection{Determining the Console Device}

The console device for a VAXstation 2000 system is the keyboard (LK201)
and monitor (VR260) connected to the video port. The keyboard inputs
commands at 4800 baud and the monitor displays output from the video
circuits.

The console device for a MicroVAX 2000 system is the terminal (VT220 or
similar terminal) connected to connector 1 on the DEC423 converter. The
terminal operates at 9600 baud.

\subsection{Entering Console Mode}

Console mode is entered any time the CPU halts. The CPU can be halted
automatically or manually. A halt means that CPU control has passed 
control from the operating system to the console mode program in ROM. If
the system halts the CPU, then the console mode program checks the 
nonvolatile RAM (NVR) for user-defined instructions on how to handle the halt.
If you manually halt the CPU, the system enters console mode program 
immediately without checking the NVR for instructions.

You can manually halt the CPU and enter console mode by one of the
following methods.

\begin{itemize}
\item HALT button -- Press the halt button. It is located next to the printer port
on the back of the system box. The \console prompt is displayed when
ready for console commands.
\item BREAK key -- Press the BREAK key on the diagnostic console device
that is connected to the printer port with the BCC08 cable. The \console
prompt is displayed when ready for console commands.
\end{itemize}
The system automatically halts the CPU for the following reasons.
\begin{itemize}
\item After power-up testing -- If the default recovery action is halt, the system
automatically halts the CPU and enters console mode after power-up
tests are complete. See Section 2.5.4 for information on setting the
default recovery actions.
\item For a boot failure -- If the system fails to boot properly, the system 
automatically halts the CPU and enters console mode. See Section 2.5.2
for information on setting the default boot device.
\item On a system error -- If the CPU detects a severe corruption of its oper-
ating environment, it halts and reads the default recovery action in the
NVR. The default recovery action can be restart, boot, or halt. When
it is restart and the restart fails, then the system automatically tries to
boot the operating system software. If the boot fails, the CPU halts and
enters console mode. When the default recovery action is boot and the
boot fails, the CPU halts and enters console mode. When the default re
covery action is halt, the CPU unconditionally halts and enters console
mode.
\end{itemize}

One other way to halt the CPU is when the operating system software 
executes a halt instruction. The CPU then reads the default recovery action in
the NVR and acts on it as described above.

\subsection{Exiting Console Mode}

Console mode is exited by typing one of the following console commands.

\begin{itemize}
\item BOOT -- This command initializes the CPU and boots the operating 
system software from the device specified. If no device is specified, the
system searches each on-line device until the operating system software
|s found. The boot command starts the system similar to when power
is turned on except that the power-up tests are not run. If the system
attempts to boot over the net (ESA0) and no software is available on
another node, the system keeps looking for the software indefinitely.
If a boot message for the operating system software does not appear
shortly after the Ethernet boot message (ESA0) is displayed, then you
must press the halt button to abort the Ethernet boot. If you still need
to boot over the Ethernet, make sure the node with the operating sys-
tem software is operating normally and the software is loaded. Run the
Ethernet loopback Utility (TEST 90) to check the networking capability
of the system if the Ethernet boot continues to fail.

When a boot is invoked using this boot command, you can specify
several boot command flags by bit encoding the flags in a flag word
specified with the /R5: qualifier. These command flags are listed in
Paragraph 2.5.3.

\item CONTINUE -- This command instructs the CPU to continue the 
operating system software at the address contained in the program counter
(PC). This command starts up the operating software where it was halted
provided no console test commands were run. Running a test command
alters the PC and memory so that the operating system software cannot
be started properly by entering the continue command. If a test com-
mand was entered, use the boot command to start the operating system
software. The continue command is similar to the start command. The
start command can specify the address to Start the operating software
and the continue command has the operating software continue where
it left off.

\item START -- This command starts the operating system software at a 
specified address. If no address is given, the contents of the PC are used.
However, running a test command alters the PC and memory so that the
operating system software cannot be started properly by entering the
start command. If a test command was entered, use the boot command
to start the operating system software.

\end{itemize}

\subsection{Diagnostic Console Device}

There is a diagnostic console device available on the VAXstation 2000. The
MicroVAX 2000 can use this device if the DEC423 Converter is removed.

The diagnostic console device can be utilized by connecting a terminal (such
as the VT100 or VT220) to the printer port with the special BCC08 cable.
The terminal operates at 9600 baud. Field service technicians can use this
terminal as a diagnostic tool to isolate a problem in the normal console
device.

To use this diagnostic console device you must turn off power, connect the
BCC08 cable to the printer port and terminal, and then turn power back on.
The diagnostic console device now controls the system. The normal console
monitor displays video test patterns on the VAXstation 2000 when the video
circuits are tested. The normal console keyboard does not operate.

\subsection{Console Commands}

Console commands are listed in \hyperlink{appendix.b}{Appendix B}.
\newpage
\section{Testing}

Testing procedures on the VAXstation 2000 and the MicroVAX 2000 systems
are almost identical. The differences between the systems are explained
where applicable. All diagnostic tests are ROM-based and testine is done
while in console mode. Tests are executed in either of two ways:

\begin{itemize}
\item Automatically -- When power is turned on the power-up tests begin.
\item Manually -- By entering one of the console test commands on the 
console terminal. See \hyperlink{appendix.a}{Appendix A} for a complete listing of the test commands.
\end{itemize}

\subsection{Power-up Tests}

Power-up tests run each time the system power is turned on. Power-up
testing consists of a sequence of tests executed for each device installed in
the system. The test number of each device is listed on the power-up screen
display as the device is tested. \figref{2-1} shows an example of the power-
up screen display. The first line indicates the CPU name (KA410-A) and the
ROM version (V1.0). The test numbers are listed next in descending order
from the first test, F, to the final test, 1. TEST F will have an underscore
after it on the MicroVAX 2000 to indicate that TEST F was not run. Note in
\figref{2-1} that tests 4, 3, 2, and 1 have an underscore (\_) immediately after
them. This underscore indicates that there is no option device installed for
that test; thus, no tests are done. TEST F has an underscore after it on the
MicroVAX 2000 systems because the monochrome video circuits are not
used by the MicroVAX 2000. No other test numbers can have underscores
after them. An asterisk (*) after TESTS 4 through 1 indicate that an option is
installed, but its ROM is destroyed and the Option device must be replaced.
Only TESTS 1 through 4 can have an asterisk after them. \figref{2-2} lists
the symbols that can appear in between the tests and what they indicate.

\begin{ttfig}{2-1}{Example of Power-up Tests Screen Display}
KA410-A V1.0
F...E...D...C...B...A...9...8...7...6...5...4_..3_..2_..1_..
\end{ttfig}

\figref{2-2} lists the definitions of the symbols that appear between the test
numbers in the power-up test countdown.

\begin{ttfig}{2-2}{Power-up Symbols Defined}
    ...  Device tested successfully or has a soft error
    ?..  Device has a hard error
    _..  Device not installed or not tested
    *..  Device installed but its ROM is destroyed
\end{ttfig}

If any hard errors (errors that indicate the device must be replaced for
proper operation) are found during power-up testing, a question mark is
placed after the failing test number during the countdown sequence. An
error summary of all errors detected is listed after the power-up sequence
is complete. Two question marks in the error summary indicate a hard error.
Error codes that indicate the status or soft errors do not put a question mark
after the failing test number in the sequence, but do list the error code in the
error summary. \figref{2-3} shows the power-up screen display with a hard
error found in TEST F and a soft error found in TEST E. The error summary
for each failed device is displayed before the boot sequence is started.
However, the screen usually scrolls so fast when the system starts to boot
that you may not be able to see what the error summary contained (if there
was an error summary). To see what errors the power-up tests found, press
the halt button and enter TEST 50 on the console terminal. TEST 50 is the
command for bringing up the configuration table. The configuration table
is created during power-up testing. This configuration table contains all of
the error codes listed in the power-up error summary as well as error codes
for all devices installed in the system. The error codes in the configuration
table are updated every time self-test is run. See Paragraph 2.5.1 for an
explanation of how to use the configuration table.

Each error summary consists of one or two question marks, a test number,
the ID number of the failed device, and an eight-digit error code. For ex-
ample, in \figref{2-3}, the first line of the error summary shows a hard error
for TEST F, a device ID number of 00B0, and an error code of 0001.F002.
The second line shows a soft error for TEST E, a device ID of 0040, and an
error code of 0000.0005. \hyperlink{subsubsection.2.3.1.1}{Section 2.3.1.1} describes the error codes.
\newpage
\begin{ttfig}{2-3}{Example of Power-up Tests Screen Display with Errors.}
KA410-A V1.0

F?..E...0...C...B...A...9...8...7...6...5...4_..3_..2_..1_..

?? F  00B0  0001.F002
 ? E  0040  0000.0006
\end{ttfig}

If there is a fatal error in the NVR during power-up testing, the system stops
testing the other devices and displays ?14 TOY ERR on the screen. When
this happens, the only way to determine the cause of the problem is by
viewing the LEDs on the keyboard. One of the LEDs will be lit to indicate
the failing module. \tabref{2-2} lists the LEDs and which module has failed.

\begin{tbl}{2-2}{Keyboard LEDs Defined}{l l}
\textbf{Keyboard LED} & \textbf{Failing module}\\
\hline
Hold Screen & System module \\
Lock & Not used \\
Compose & ThinWire Ethernet option module \\
Wait & Not used \\
\end{tbl}

Once power-up testing is complete and no fatal or hard errors are found, the
system boots the operating system software. Both the VAXstation 2000 and
the MicroVAX 2000 boot the operating system software the same. The only
difference between the two is that the VAXstation 2000 clears the console
screen before it boots and the MicroVAX 2000 does not. If a default boot
device is loaded in the NVR, the system boots off of that device. If no
default device is loaded in the NVR, the system searches every on-line
storage drive for the operating software. DUA2 is searched first if a floppy
diskette is loaded. Otherwise, it is not searched at all. The hard disks are
searched next, DUA0 then DUA1. MUA0, the tape drive, is checked after
the hard disk drives if it is installed and a cartridge is loaded. Finally, the
system searches the Ethernet network for the software and ESA0 is listed
on the screen. The system continues to search the Ethernet network until
the operating system software is found.
\newpage
\subsubsection{Power-up Test Error Codes}

The power-up test error codes indicate status and/or error information. Any
errors found by power-up tests are listed in the error summary after the
power-up test countdown sequence. This summary, if any, gives you a brief
summary of the errors. \tabref{2-3} lists the test numbers and the devices that
are tested during that particular test. To look at the complete list of devices
and the status of that device, you must display the configuration table. The
configuration table lists every device in the system and also lists the results
of the self-test and power-up tests and is updated each time self-test is run.
The error codes for each device in the configuration table are explained in
the troubleshooting section for that individual device. Remember that the
configuration table contains the results of the self-test and power-up tests
and not the results of the system exerciser. \figref{2-4} shows an example
of the configuration table and for an explanation of the configuration table,
see \hyperlink{subsection.2.5.1}{Paragraph 2.5.1}.


\begin{ttfig}{2-4}{Example of the Configuration Table}
>>> TEST 50

KA410-A V1.0
ID 08-00-2B-02-CF-A4

?? MONO       0001.F002
 ? CLK        0000.0005
   NVR        0000.0001
   DZ         0000.0001
       00000001 00000001 00000001 00000001 00000001 000012A0
   MEM        0002.0001
       00200000
   MM         0000.0001
   FP         0000.0001
   IT         0000.0001
   HDC        1710.0001
       000146B8 00000000 00000320
   TPC        0202.0001
       FFFFFF03 01000001 FFFFFF06 FFFFFF05 FFFFFF05 FFFFFF05 ...
   SYS        0000.0001
   NI         0000.0001 V1.0
>>>
\end{ttfig}
\newpage
The most common good error code is 0000.0001. There are, however, some
devices that use the first four digits in the error code to indicate the status
of the device and the last four digits to indicate the error found on the
device. The memory (MEM) error code, for instance, contains 0002.0001
which indicates two megabytes of memory is available (0002.) and no error
found (.0001). On devices like these, the last four digits always indicate
.0001 as a good (non-error) indication.

Some error codes indicate no error at all and give a status of the device
such as the clock (CLK) which shows that the date and time has not been
set. This is not an error, just a status of the clock circuits.

Any error code other than 0000.0001 on the MONO, MM, FP, IT, or SYS
devices indicates a hard error and that device must be replaced for proper
operation of the system. The other devices such as CLK, NVR, DZ, MEM,
HDC, TPC, and NI may have a status or a soft error message in the error
codes and may still operate normally.

See the troubleshooting procedures section (Paragraph 2.4) for each device
to determine whether or not the error code indicates a fault or a status for
the device.

\subsection{Self-test}

Self-test allows you to test every device again individually, a few at a time,
or all of them sequentially just like power-up tests. To individually test a
device, enter TEST \# where \# is the test number of the device you want
tested. \tabref{2-3} lists the test numbers and the devices tested by those
numbers. \figref{2-5} shows an example of running self-test successfully on
the disk controller.

\begin{ttfig}{2-5}{Example of Running Self-test on the Disk Controller}
>>> TEST 7
  7...
>>>
\end{ttfig}

\newpage

To test a group of devices, enter TEST followed by the test number of the
first device to be tested and then the test number of the last device to be
tested. \figref{2-6} shows an example of testing a group of devices. In \figref{2-6}, 
all tests between C and 4 are tested successfully. Note that you cannot
pick and choose which devices to test between C and 4, all tests between
C and 4 are tested when entered as a group.

\begin{ttfig}{2-6}{Example of Running a Series of Self-tests}
>>> TEST C 4
  C...B...A...9...8...7...6...5...4_..
>>>
\end{ttfig}

To test all devices, enter TEST F 1. The MicroVAX 2000 skips over the
MONO video test (TEST F) since it does not use the video circuits.

\begin{tbl}{2-3}{Self-test Commands}{p{2cm} p{8cm}}
\raggedright\textbf{Test\newline Number} & \textbf{Device\newline Tested} \\
\hline
1 & Option module (Network Interconnect module) (NI) \\[0.5em]
2 & Option module (not available) \\[0.5em]
3 & Option module (not available) \\[0.5em]
4 & Option module (not available) \\[0.5em]
5 & Interrupt Controller and ThinWire Ethernet ID ROM (SYS) \\[0.5em]
6 & Tape Controller. (TPC) \\[0.5em]
7 & Disk conroller. (HDC) \\[0.5em]
8 & Interval timer. (IT) \\[0.5em]
9 & Floating point unit. (FP) \\[0.5em]
A & Memory management unit. (MM) \\[0.5em]
B & Memory. (MEM) \\[0.5em]
C & DZ Controller. (DZ) \\[0.5em]
D & Non-volatile RAM. (NVR) \\[0.5em]
E & Time-of-year dock. (CLK) \\[0.5em]
F & Base video (MONO) (VAXstation 2000 only) \\[0.5em]
\end{tbl}

\subsubsection{Self-test with Loopback Connectors}

Customer mode self-test does not test the drivers or the lines of the serial
line conroller (DZ) since loopbacks are not used. Run self-test in field 
service mode to test the DZ drivers by installing the loopback connectors on
the back of the system. Follow one of the procedures below.

To test the DZ on VAXstation 2000, install a loopback (p/n 29-24795) on
the 25-pin communication port and a loopback (p/n 29-24794) on the 9-pin
printer port. Run TEST C. You cannot use loopback connectors if you are
using the diagnostic console device with the BCC08 cable on the printer
port since there is no loopback connector for the video port.

To test the DZ on MicroVAX 2000, install a loopback (p/n 29-24795) on the
25-pin communication port and install an MMJ loopback on both ports 2
and 3. Run TEST C.

\textbf{NOTE:} \textit{The ThinWire Ethernet port on the back of the 
system box must be terminated properly when running diagnostics on the 
network option (TEST 1) othetwise an error code of 0000.7001 or greater 
is listed in the configuration table.}

\subsubsection{Self-test Error Codes}

\figref{2-7} shows how an error is displayed if found during self-test. This
example shows an error on the disk controller during self-test. The 84 FAIL
indicates an error was found on the device tested. You must display the
configuration table (TEST 50) after self-test is complete to see the error code,
if there is an error during self-test, since the error codes do not appear on
the screen. The configuration table lists every device in the system, fists the
results of the self-test and power-up tests, and is updated each time self-test
is run. The error codes for each device in the configuration table are 
explained in the troubleshooting section for that individual device. Remember
that the configuration table contains the results of the self-test and power-up
tests and not the results of the system exerciser. See Paragraph 2 5 1 for an
explanation of the configuration table.

\newpage 
\begin{ttfig}{2-7}{Example of a Self-test Error on the Disk Controller}
  >>> TEST 7
    7?..
    84 FAIL
  >>>
(You must display the configuration table to see the error code)
\end{ttfig}

\newpage

\subsection{System Exerciser Diagnostics}

The system exerciser simulates a worst-case situation test for each device
and checks how the device operates under these conditions. It does not use
the configuration table to list its results of the tests. The system exerciser
has a separate display that appears on the screen as the exerciser is running.
Any errors found are displayed in the exerciser display. When examining
the exerciser display, a single question mark in the far left column indicates
a soft error, a double question mark indicates a hard error, and the absence
of question marks indicate success. \figref{2-8} shows an example of the
exerciser display.

The system exerciser exercises most of the devices. However, some devices
such as the memory management unit (MM) and the interval timer (IT) are
tested through the testing of other devices and are not displayed. Of the
devices the exerciser does exercise, it runs each one sequentially until all
have been run once, then it runs them all at the same time (worst-case).
This type of testing usually finds any intermittent failures.

The system exerciser has two modes: customer mode and field service
mode. Customer mode system exerciser (TEST 0) does not use loopback
connectors and does not fully test all of the devices. Field service mode sys-
tem exerciser requires loopback connectors installed and removable media
from the maintenance kit inserted and loaded. You must initialize the floppy
diskette in the maintenance kit with a special diagnostic key so the exerciser
can perform write tests on the RX33. Refer to Paragraph 2.5.11 for more
Information on creating the special diagnostic keys. Once the floppy has
been initialized, load it into the RX33 (if a full read/write test of the RX33 is
necessary) before you run the system exerciser in field service mode. You
must also initialize the COMPACTape cartridge in the maintenance kit with
a special diagnostic key so the exerciser can perform read/write tests on
the TK50. Refer to Paragraph 2.5.11 for more information on creating the
special diagnostic keys. Once the COMPACTape cartridge has been initial
ized, load it into the TK50 (if a full read/write test of the TK50 is necessary)
before you run the system exerciser in field service mode. If the RX33 or
the TK50 are not loaded with the special-key media, the system exerciser
does not do destructive writes to them and tests them the same as it does
during the customer mode system exerciser. This special-key on the media
prevents the exerciser from accidentally destroying data on the customers
floppy diskette or COMPACTape cartridge. The field service mode system
exerciser is available in a run once (TEST 101) and a run forever (TEST 102)
configuration.

\newpage

\subsubsection{System Exerciser Diagnostic Commands}
\tabref{2-4} lists the system exerciser diagnostic commands.

\begin{tbl}{2-4}{System Exerciser Diagnostic Commands}{p{0.2\textwidth} p{0.8\textwidth}}
\textbf{Test Commands} 	&	\textbf{Description of Commands} \\
\hline

0						&	Runs customer mode system exerciser. It exercises 
							each device once sequentially, then exercises them 
							simultaneously, and stops when the slowest device
							finishes its second pass. No loopback connectors 
							and no removable media required. \\

101						&	Runs field service system exerciser. It exercises 
							each device once sequentially, then exercises them 
							simultaneously, and stops when the slowest device 
							finishes its second pass. Do not stop the exerciser 
							before it finishes exercising every device twice 
							(second pass). Loopbacks and removable media required. \\

102						&	Runs field Service system exerciser. It exercises each 
							device once sequentially and then exexcises them 
							simultaneously until you enter a \keystroke{CTRL}-\keystroke{C}. 
							Note that the exerciser takes up to thirty seconds 
							to stop after you enter \keystroke{CTRL}-\keystroke{C}. Do
							not stop the exerciser until every device is exercised 
							twice (second pass). Also, do not press the halt button 
							to stop the exerciser. Loopbacks and removable media required. \\
\end{tbl}

When the exerciser is started, PRA0 is displayed and the monitor connected
to the video port on the VAXstation 2000 blinks white and black several
times while the monochrome circuits are being tested. The results of the
MONO tests are then displayed on the console screen. MicroVAX 2000
does not use the MONO circuits and, therefore, does not test them. The
DZ test results are the first to be displayed on the console screen. The rest
of the devices are then exercised one at a Urne and the results are listed
on the display. The console displays the results of each device until the
last device is finished testing. When the last device is done, the exerciser
starts running all devices together at the same time. When this happens, the
monitor connected to the video port starts blinking again until the slowest
device finishes testing. On the MicroVAX 2000, the console terminal holds
the first pass display until the slowest device is done testing then displays a
new exerciser display. The console screen (VAXstation 2000 and MicroVAX
2000) displays the results of each device tested. This display stays on the
screen for about 10 seconds, then the exerciser starts running all devices
together again if TEST 102 was entered. If TEST 0 or TEST 101 was entered,
the exerciser stops after the slowest device is done testing. The halt message
is displayed when the exerciser is stopped.

To run the field service mode system exerciser on VAXstation 2000, install a
loopback (p/n 29-24795) on the 25-pin communication port and a loopback
(p/n 29-24794) on the 9-pin printer port. Insert and load the floppy diskette
and TK50 COMPACTape cartridge from the maintenance kit if the system
has these devices installed. Run the exerciser by entering TEST 101 or TEST
102. If you are using the diagnostic console device, you will get errors on
the DZ line since there is no loopback for the video port.

To run the field service mode system exerciser on MicroVAX 2000, install
a loopback (p/n 29-24795) on the 25-pin communication port and install an
MMJ loopback on both ports 2 and 3. Insert and load the scratch floppy
diskette and scratch TK50 COMPACTape cartridge from the maintenance
kit if the system has these devices installed. Run the exerciser by enter
ing TEST 101 or TEST 102. If the scratch floppy or TK50 cartridge is not
installed, the system tests those devices the same as in customer mode.

\subsubsection{System Exersiser Error Codes}

\figref{2-8} shows an example of the system exerciser display while running
the system exerciser in field service mode. Customer mode gives the same
display, but with a CU in place of the FS on the top of the display.

\begin{ttfig}{2-8}{System Exerciser Display Example}
KA410-A V1.0          01     FS

   F  00B0    MONO    0000.0001      2      0 00:00:55.01
   C  0080    DZ      0000.0001      5      0 00:01:28.03
   B  0010    MEM     0175.0001      2      0 00:02:03.07
   7  0090    HDC     2000.0001      4      0 00:02:08.58
?                     0700.7091             0 00:02:08.58
??                    1002.0051             0 00:01:03:45
                      2500.0001
   6  00A0    TPC     1000.0001      9      0 00:02:44.04
                      1300.0001
?? 1  OOCO    NI      0000.7004      3      0 00:04:46.32

>>>

  (The error codes are defined in the troubleshooting
   procedures for each individual device)
\end{ttfig}

The first line indicates the CPU name (KA410-A), the ROM version (V1.0),
the ROM Status (01 -- the ROM is corrupted if this is anything other than
01), and the mode of the system exerciser (CU for customer and FS for field
service mode). The next several lines list information on the devices that
are exercised by the system exerciser.

The first column of the display lists the test number (F,C,B,7,6,1). TESTS 4
through 1 are option modules and these options may not be installed. They
are listed in the display only if they are installed. The second column lists
the device identifier (00B0, 0080, 0010,...). Next is the mnemonic for each
device (MONO, DZ, MEM,...) followed by the error code (0000.0000) for
that device (the HDC and TPC devices have additional error codes for each
drive). After the error code is the number of times the device was tested
followed by the time of the last pass the exerciser made on that device.
The time is in the format of days hours:minutes:seconds.hundredths of
seconds. Question marks identify hard (??) and soft (?) errors. See the
troubleshooting procedures (\hyperlink{section.2.4}{Paragraph 2.4}) for 
the device that has an error detected with it.

\section{Troubleshooting Procedures}

Each section below contains troubleshooting information for the device 
indicated. Find the section below with the mnemonic (such as MONO or TPC)
that you want to troubleshoot. Once the section is located, read through the
procedures to familiarize yourself with the testing of the device.

\subsection{MONO -- Monochrome Video Troubleshooting Procedures (VAXstation 2000 only)}

You can troubleshoot the monochrome circuitry on the system module
in either customer mode or field service mode. Both modes test the
monochrome circuits the same. This procedure does not troubleshoot the
video monitor, just the video circuits on the system module.

\subsubsection{Self-test}

To run self-test on MONO, enter TEST F. The monitor connected to the
video port blinks white and black several times. This is normal. The results
of the self-test are displayed when self-test is complete. Any error code
other than 0000.0001 indicates a fault in the monochrome video circuits.
You must replace the system module to fix this problem.

\subsubsection{System Exerciser}

To run the system exerciser, enter TEST 0 for customer mode or TEST 101
for field service mode. The monitor connected to the video port blinks white
and black several times while the monochrome circuits are being tested. The
results of the MONO tests are displayed on the console screen after they are
tested for the first time, then displayed again after each concurrent testing.
Any error code other than 0000.0001 indicates a fault in monochrome video
circuits. You must replace the system module to fix this problem. This
device is not exercised on the MicroVAX 2000 system since MicroVAX 2000
does not use video monitors.

\subsection{CLK -- Time-of-Year Clock Troubleshooting Procedures}

You can troubleshoot the CLK circuitry in either customer mode or field
service mode. Both modes test the CLK circuits the same.

\subsubsection{Self-test}

To run self-test, enter TEST E. Any error code other than 0000.0001 
indicates a fault in the CLK circuits. An error code of 0000.0005 indicates that
the system time is not set. Refer to the operating system software 
documentation to set the time. If any other error code appears in the CLK error
display, you must replace the system module.

The system exerciser does not display the Status of the CLK circuits. The
CLK circuits are not directly tested but are tested through the testing of
other circuits.

\subsection{NVR -- Non-Volatlle RAM Troubleshooting Procedures}

You can troubleshoot the NVR circuitry in either customer mode or field
service mode. Both modes test the NVR circuits the same.

\subsubsection{Self-test}

To run self-test, enter TEST D. Any error code other than 0000.0001 in
dicates a fault in the NVR. An error code of 0000.0005 indicates that the
battery charge is below the normal voltage level. If the battery is low, al-
low the system to charge the battery for five minutes and then run TEST D
again. If the error code is still 0000.0005, replace the battery. A charging
time of 20-25 hours is needed to fully charge the battery. Leaving the sys-
tem powered up charges the battery. If any other error code appears in the
CLK error display, you must replace the system module.

The system exerciser does not display the status of the NVR circuits. The
NVR circuits are not directly tested but are tested through the testing of
other circuits.

\subsection{DZ -- Serial Line Controller Troubleshooting Procedures}

You can troubleshoot the DZ in either customer mode or field service mode.
The difference between the two modes is that field service mode tests the
serial line drivers on the system module and customer mode does not.

\subsubsection{Self-test}

To run customer mode self-test on either the VAXstation 2000 or the 
MicroVAX 2000, enter TEST C. Loopback connectors must not be installed
when in customer mode. Loopback connectors must, however, be installed
to run field service mode diagnostics. Any error code other than 0000.0001
indicates a fault in the DZ controller. You must replace the system module
to fix the problem. If you are using the diagnostic console device, any error
code other than 0000.4001 indicates a fault in the DZ controller.

\textbf{VAXstation 2000}

To run self-test in field service mode on VAXstation 2000, install a loopback
(p/n 29-24795) on the 25-pin communication port and a loopback (p/n 29-
24794) on the 9-pin printer port. Enter TEST C. If the error code is not
0000.0001, check to see if the loopbacks are still connected. Reconnect
them, if necessary, and test again. Replace the system module if the error
still exists. Loopbacks cannot be used to test the DZ controller when the
console device is the diagnostic console with the BCC08 cable since there
is no loopback for the video port.

\textbf{MicroVAX 2000}

To run self-test in field service mode on MicroVAX 2000, install a loop
back (p/n 29-24795) on the 25-pin communication port and install an MMJ
loopback on both ports 2 and 3. Enter TEST C. If the error code is not
0000.0001, check to see if the loopbacks are still connected. Reconnect
them, if necessary, and test again. Replace the system module if the error
still exists.

\subsubsection{System Exerciser}

To run the customer mode system exerciser on either the VAXstation 2000
or the MicroVAX 2000, enter TEST 0. The results are displayed on the video
screen as the tests are completed.

\textbf{VAXstation 2000}

To run the field service mode system exerciser on VAXstation 2000, install a
loopback (p/n 29-24795) on the 25-pin communication port and a loopback
(p/n 29-24794) on the 9-pin printer port. Enter TEST 101. If you are using the
diagnostic console device, you will get errors on the DZ line since there is
no loopback for the video port. Any error code other than 0X00.0001, where
X is the serial line being used for the console device, indicates a fault in
the DZ controller. You must replace the system module to fix the problem.
If the console device is connected to the video port on VAXstation 2000,
then you will see a 0 (zero) in the X position. If the console device is the
diagnostic console device with the BCC08 cable on the printer port, then
you will see a 3 in the X position. If the error code is not 0X00.0001, check
to see if the loopbacks are still connected. Reconnect them, if necessary,
and test again. Replace the system module if the error still exists.

\textbf{MicroVAX 2000}

To run the field service mode system exerciser on MicroVAX 2000, install
a loopback (p/n 29-24795) on the 25-pin communication port and install an
MMJ loopback on both ports 2 and 3. Enter TEST 101 or TEST 102. Any
error code other than 0X00.0001, where X is the serial line being used for
the console device, indicates a fault in the DZ controller. You must replace
the system module to fix the problem. With the console device connected to
port 1 on the MicroVAX 2000, you will see a 0 (zero) in the X position. If the
error code is not 0X00.0001, check to see if the loopbacks are still connected.
Reconnect them, if necessary, and test again. Replace the system module
if the error still exists. If the error still exists after replacing the system
module, replace the DEC423 converter on the back of the system box.

\subsection{MEM -- Memory Troubleshooting Procedures}

You can troubleshoot memory in either customer mode or field service
mode. Both modes test the MEM circuits the same. These procedures are
for both VAXstation 2000 and MicroVAX 2000.

\subsubsection{Self-test}

To run self-test, enter TEST B. Any error code other than 000X.0001, where
X is the size of memory (megabytes) in the system, indicates a fault in the
memory circuits. The problem could either be with the system module or
the memory Option module. To determine which module is at fault go to
\hyperlink{subsubsection.2.5.1.2}{Paragraph 2.5.1.2}.

\subsubsection{System Exerciser}

To run the system exerciser, enter TEST 0 for customer mode or TEST 101
for field service mode. \figref{2-9} shows what the MEM system exerciser
error code indicates. The status portion of the code indicates the number
of pages tested during the last test pass (1 page = 512 bytes) if there were
no errors found.
\newpage
\begin{ttfig}{2-9}{Example of MEM System Exerciser Error Code}
.
.
.
B  0010    MEM     0175.0001     2    0 00:02:03.07 
.                  |  | |  |
.                  `--' `--'
.                   |    |
                    |    `-- Error code. 0001 = GOOD
                    |
                    `------- Status code. Number of
                             pages of memory tested
                             when error code is 0001.
\end{ttfig}

An error code of .0001 indicates no errors. If errors were detected during
the last test pass, the error portion of the code contains the error code.
\tabref{2-5} lists the MEM system exerciser errors and shows which module
is causing the error.

\begin{tbl}{2-5}{MEM System Exerciser Error Codes}{l l}
\textbf{Error Codes} & \textbf{Definition}\\
\hline
0001.001F   &   Compare error on the system module\\
0002.001F   &   Compare error on the Option module\\
0001.002F   &   Parity error on the system module\\
0002.002F   &   Parity error on the Option module\\
\end{tbl}

\subsection{MM -- Memory Management Unit Troubleshooting Procedures}

You can troubleshoot the memory management (MM) circuitry in either
customer mode or field service mode. Both modes test the MM circuits the
same.


\subsubsection{Self-test}

To run self-test, enter TEST A. Any error code other than 0000.0001 
indicates a fault in the memory managment circuits. You must replace the
system module to fix the problem.

The system exerciser does not display the status of the MM circuits. The
MM circuits are not directly tested but are tested through the testing of other
circuits.

\subsection{FP -- Floating Point Unit Troubleshooting Procedures}

You can troubleshoot the floating point (FP) circuitry in either customer
mode or field service mode. Both modes test the FP circuits the same.

\subsubsection{Self-test}

To run self-test, enter TEST 9. Any error code other than 0000.0001 indicates
a fault in the floating point circuits. You must replace the system module
to fix the problem.

The system exerciser does not display the status of the FP circuits. The FP
circuits are not directly tested but are tested through the testing of other
circuits.

\subsection{IT -- Interval Timer Troubleshooting Procedures}

You can troubleshoot the interval timer (IT) circuitry in either customer
mode or field service mode. Both modes test the IT circuits the same.

\subsubsection{Self-test}

To run self-test, enter TEST 8. Any error code other than 0000.0001 indicates
a fault in the timing circuits. You must replace the system module to fix the
problem.

The system exerciser does not exercise the IT circuits.

\subsection{HDC -- Disk Drives and Controller Troubleshooting Procedures}


You can troubleshoot the disk controller and drives using self-test and the
system exerciser. These diagnostics test the disk controller on the system
module and also test the drives connected to the controller. There is a max-
imum number of three drives that can be supported in either a VAXstation
2000 or a MicroVAX 2000. The three drives are labelled DUA0, DUA1, and
DUA2. DUA0 will always be the hard disk drive (RD) located in the system
box. DUA0 can be a full-height or a half-height drive. DUA1 will always be
the hard disk drive (RD) located in the expansion box. DUAI communicates
to the system module through port B on the expansion adapter; thus allow-
ing you to isolate DUA1 during testing, if necessary, without opening the
system box. DUA2 will always be the half-height floppy disk drive (RX33)
and will only be located in the system box. If DUA0 is a full-height drive,
then a floppy disk drive cannot be installed because of lack of space in the
system box. The disk controller labeis any drive off-line that is not installed.
It also labeis DUA2 off-line if a floppy diskette is not properly loaded.

\subsubsection{Self-test}

To run self-test on the HDC, enter TEST 7. Self-test gives a quick status
of the disk controller on the system module and the drives. You can run
self-test in either customer mode or field service mode since both modes
test these devices the same. The error code for the disk controller (HDC in
the configuration table) contains the test results of the disk controller and
the status of the three drives. \figref{2-10} shows how the error code is
broken into five segments: status of DUA2, DUA1, DUA0, tape controller,
and the error code if a hard error is found. The power-up error code is the
same as the self-test error code.

\begin{ttfig}{2-10}{HDC Power-up and Self-test Error Code}
7 0090 0000.0000
       |||| |  |
       |||| `--'
       ||||  |
       ||||  `---> These four digits echo the first four digits
       ||||        if a hard error is found on the disk
       ||||        controller. Otherwise, 0001 = Good.
       ||||
       |||`------> Status of disk controller on system module.
       |||         0 = Good.
       |||
       ||`-------> Status code for DUA0, listed in Table 2-6.
       ||
       |`--------> Status code for DUA1, listed in Table 2-6.
       |
       `---------> Status code for DUA2, listed in Table 2-6.

              DUA0 is the hard disk drive in the system box.
              DUA1 is the hard disk drive in the expansion box.
              DUA2 is the floppy drive in the system box.
\end{ttfig}

Each drive has the same set of error codes. These codes are listed in 
\tabref{2-6}. All odd-numbered error codes are soft errors or a status. All 
even-numbered error codes (including A and F) are hard errors. The last four
digits of the error code repeat the first four digits if a hard error is found on
the disk controller. Otherwise, the last four digits contain 0001 to indicate
no errors or soft errors.

\begin{tbl}{2-6}{Power-up and Self-test Error Codes for each Dlsk Drive}{p{0.2\textwidth} p{0.75\textwidth}}
\textbf{Error Codes} & \textbf{Description of error codes for each disk drive}\\
\hline
1	&	Good -- No error for this drive.\\

2	&	Drive select error. Disk controller or the drive failed. Replace the system
		module first. Replace the drive if the problem is not fixed after replacing
		the system modile. \\

3	&	Read during read test error. The disk or diskette may not be formatted. Run
		the disk verifier to chekc out the disk. Copy the disk fata onto another disk
		or to another system over the net if you have to reformat the drive. \\

4	&	Read after write error. Drive failed. Replace the drive first. Replace the
		system module of the problem is not fixed after replacing the drive. \\

5	&	Invalid UIB (DUA0 and DUA1 only). Disk needs formatting or the disk is
		not a Digital disk. Run the disk verifier to check out the disk. Copy the
		disk data onto another disk or to another system over the net if you have
		to reformat the drive. \\

6	&	Drive failed to restore. Drive failed. Replace the drive first. Replace the
		system module if the problem is not fixed after replacing the drive. \\

7	&	Off-line — No drive installed, no floppy diskette loaded in DUA2, or DUA1
		(in expansion box) is not turned on. \\

8	&	Drive not done error. Drive failed. Replace the drive first. Replace the
		system module if the problem is not fixed after replacing the drive. \\

9	&	Invalid Status from controller. Disk controller or diskette failed. If DUA0
		or DUA1, replace the system module first then replace the disk drive if the
		problem is not fixed after replacing the system module. If DUA2, replace
		the floppy diskette first or save the data on it and reformat it. If DUA2 and
		the floppy diskette is not the problem, replace the system module and then
		replace the floppy disk drive if replacing the system module did not fix the
		problem. \\

A	&	Drive select timeout error. Drive failed. Replace the drive first. Replace
		the system module if the problem is not fixed after replacing the drive. \\

F	&	Untested -- Drive was not tested because of a hard error found on the disk
		controller. For example, FFF8.FFF8 indicates an error on the disk controller
		and no drives were tested. Replace the system module. \\
\hline
\end{tbl}

If any error (except 7 and F) appears for any drive, check the drive for
power and check the cables for a good connection. If the error still exists,
the problem is either in the drive, in the cables, in the system module, or in
the disk interface module located in the expansion adapter. If, for example,
you replace one of the disks to fix an error code and the error still exists,'
replace the system module. If the system module does not fix the problem
replace the disk interface module.

\subsubsection{System Exerciser}

Start the system exerciser by entering TEST 0 for customer mode or install
the 25-pin loopback on the communications port and enter TEST 101 for
field service mode. The customer mode system exerciser does not exercise
the disks as thoroughly as the field service exerciser. The field service
system exerciser performs a complete read/write test on all drives and also
performs a data transfer test between the disk controller and one of the
drives.

The results of the system exerciser are displayed on the screen after the
first test pass of each device tested and again after all devices have been
run concurrently. \figref{2-11} shows the system exerciser display for the
disk controller (HDC). There is one line for the controller Status and one
line for each drive connected to the controller. A drive that is not installed
or is off-line is not listed in the display. For example, no diskette in DUA2
or an unformatted diskette in DUA2 labeis DUA2 as off-line and no display
for DUA2 is listed. If there are two question marks on the controllers line,
replace the system module. If there is a single question mark on the drive's
line, there is a soft error in the drive and the drive may operate normally.
Two question marks on the drive's line indicate a hard error in the drive
or an error in the controller. You must replace one or both to fix the error.
Two question marks for DUA2 may also indicate bad media on the floppy
diskette.

\begin{ttfig}{2-11}{Example of System Exerciser Display for the Disk Controller}
.
.
.
7  0090    HDC    1000.0001     2     0 00:02:09.47
                  0700.0001
                  1700.0001
                  2500.0001
.
.
.
\end{ttfig}

\tabref{2-7} lists the erorr codes for the disk controller's line and \tabref{2-8}
lists the error codes for the drives.

\begin{tbl}{2-7}{HDC Disk Controller System Exerciser Error Codes}{p{0.1\textwidth} p{0.3\textwidth} p{0.5\textwidth}}
\textbf{Error Codes} & \textbf{Possible Cause} & \textbf{Corrective Action}\\
\hline

X000.0001	&	The X indicates the drive used for the data transfer test &
	0 = DUA0, 1 = DUA1, 2 = DUA2, and F = no data transfer test was done. \\

0X00.0001	&	Data transfer error if X is anything other than zero &
	Replace the system module. If the error still exists after replacing the system module, replace the
	drive that was used for the data transfer test. Always replace the drive's device electronics board
	(hard disk drives only) before replacing the whole drive. \\
 
00XX.0001	&	The XX indicates the number of errors detected during the data transfer test.
				Make note of the drive used for the data transfer test. &
	Run the exerciser again. Was the same drive used for the transfer test?

	If yes, and the number of transfer errors are the same (or dose to the same), replace system module. 
	Replace the drive that was used for the data transfer test if replacing the system module did
	not fix the problem.

	If no, and the number of transfer errors are zero, replace the drive that got errors during the data
	transfer test.  Always replace the dxive's device electronics board (hard disk drives only) before
	replacing the whole drive. \\
	
0000.XXX1	&	Controller error. &
	If XXX is anything other than 000 (three zeros), replace the system module.\\

\end{tbl}

\begin{tbl}{2-8}{HDC Dlsk Drive System Exerciser Error Codes}{p{0.1\textwidth} p{0.25\textwidth} p{0.55\textwidth}}
\textbf{Error Codes} & \textbf{Possible Cause} & \textbf{Corrective Action}\\
\hline

X000.0001	&	The X position indicates the drive that this error code is for. &
	0 = DUA0, 1 = DUA1, and 2 - DUA2. 
\\
0X00.0001	&	The X position indicates the drive Status. &
	The X position indicates drive status as listed below.

	DUA0 and DUA1:\newline
		\hspace*{1em}7 writeable, formatted, UIB and RCT ok\newline
		\hspace*{1em}5 writeable, formatted, no UIB and RCT\newline
		\hspace*{1em}4 writeable, unformatted, no UIB and RCT\newline
		\hspace*{1em}3 non-writeable, formatted, UIB and RCT ok\newline
		\hspace*{1em}1 non-writeable, formatted, no UIB and RCT\newline
		\hspace*{1em}0 non-writeable, unformatted, no UIB and RCT

	DUA2:\newline
		\hspace*{1em}5 writeable and formatted\newline
		\hspace*{1em}4 writeable and unformatted\newline
		\hspace*{1em}1 non-writeable and formatted\newline
		\hspace*{1em}0 non-writeable and unformatted
\\
00XX.0001	&	The XX position indicates the drive error count. 00 indicates no error. &
	Run the exerciser again. Note the error count for the failing drive after each pass. If the count stays
	the same or increases, replace the failing drive.  If the error still exists after replacing the drive,
	replace the system module. Always replace the drive's device electronics board (hard disk drives
	only) before replacing the whole drive.
\\
0000.XXX1	&	The XXX position indicates the error codes for the drive. 000 indicates no error. &
	If anything other than zeros (0001), replace the system module. Replace the drive if replacing the
	system module did not fix the problem. Always replace the drive's device electronics board (hard
	disk drives only) before replacing the whole drive.
\\
\end{tbl}
\newpage

\subsubsection{Troubleshooting the Hard Disk Drive Expansion Box}

Troubleshoot the hard disk in the expansion box (DUA1) using self-test and
the system exerciser. However, if the diagnostic tests indicate an error on
DUA1, perform the steps in the following procedure.

\begin{enumerate}
\item	Check to make sure the expansion box has power to it and it is switched on.

\item	Check to make sure the hard disk cable is properly connected to port
		B on the expansion adapter and the back of the expansion box.

\item	Run the tests again and if the status code shows DUA1 to be off-line,
		troubleshoot the power supply in the expansion box.

\item	Disconnect the cable from port B on the expansion adapter and run
		self-test (TEST 6). If any status other than the off-line indication (7) or
		the not tested indication (F) shows up in the status code for DUA1, then
		replace the disk controller on the system module.

\item	If the status code does show that DUA1 is off-line after disconnecting
		it from port B, check the drive select jumpers on DUA1 for proper
		positioning. Refer to \hyperlink{subsubsection.2.4.9.4}{Paragraph 2.4.9.4} 
		below for proper drive select jumper settings.

\item	If the drive select jumper is set properly, replace DUAl in the disk
		expansion box. Always replace the drive's device electronics board
		before replacing the whole drive.
\end{enumerate}

\subsubsection{Drive Select Jumper Settings on Disk Drives}

DUA0 -- Refer to \figref{2-12} to set the drive select jumper on an RD32 in
the system box. Refer to \figref{2-13} to set the drive select jumper on an
RD53 in the system box.

DUA1 -- Refer to \figref{2-13} to set the drive select jumper on an RD53 in
the expansion box.

DUA2 -- Refer to \figref{2-14} to set the drive select jumper on an RX33 in
the system box.

\fig[0.6]{2-12}{MA-0131-87}{RD32 Drive Select Jumper Setting for DUA0}
\fig[0.6]{2-13}{SHR-0126-85}{RD53 Drive Select Jumper Setting for DUA0 and DUA1}

\fig[0.7]{2-14}{SHR-0045-86}{RX33 Drive Select Jumper Setting for DUA2}

\subsection{TPC -- Tape Drive Controller Troubleshooting Procedures}

You can troubleshoot the tape drive Controller on the system module and
the tape drive controller in the expansion box using self-test or the system
exerciser. The tape drive expansion box has an internal self-test to 
troubleshoot the TK50 tape drive as described in \hyperlink{subsubsection.2.4.10.3}{Paragraph 2.4.10.3}.

These troubleshooting procedures assume that only one tape expansion box
is connected to the tape port (port A) on the expansion adapter. Although
the diagnostic firmware located in ROM supports up to seven devices on
port A of the expansion adapter, we will only cover how to troubleshoot one
tape expansion box since the VMS and ULTRIX operating systems support
only one tape expansion box on either system. Also, the tape expansion
box must be configured for ID address 1 on the tape port.

\subsubsection{Self-test}

To run self-test, enter TEST 6. Self-test gives a quick status of the tape
controller on the system module and also the tape drive expansion box
connected to the tape controller via port A of the expansion adapter. You
can run self-test in either customer mode or field Service mode since both
modes test these devices the same. Enter TEST 50 to see the results of the
self-test. The TPC error code contains the test results of the tape controller
on the system module and the tape drives connected to port A on the
expansion adapter. \figref{2-15} shows how the TPC error code is broken
into five segments: ID addresses of devices connected to the tape port,
ID addresses of the devices that tested successfully, two possible cause
indicators, and a status of the tape controller.

\begin{ttfig}{2-15}{TPC Power-up and Self-test Error Code}
6  00A0 0000.0000
        |||| ||||
        `|`| ||`|
         | | || `--> Status of tape controller on system module.
         | | ||       01 = Good.
         | | ||
         | | |`----> Possible cause indicator.
         | | |        0 = Good.
         | | |        1 = Error most likely on system module.
         | | |        2 = Error most likely in expansion box.
         | | |        3 = Error could be in either location.
         | | `-----> Possible cause indicator.
         | |          0 = Good.
         | |          1 = Retest TPC again.
         | |          2 = Error most likely in expansion box.
         | |          3 = Combination of 1 and 2.
         | |          4 = DMA and interrupts not tested.
         | |          5 = Combination of 1 and 4.
         | |          6 = Combination of 2 and 4.
         | |          7 = Combination of 1, 2, and 4.
         | |          
         | `-------> Status of the tape expansion box. The
         |           ID address of the expansion box is
         |           displayed here if it tested
         |           diccessfully. These two digits should
         |           be the same as the first two digits.
         |            00 = No box connected to port A or
         |                 no box tested successfully.
         |            02 = ID Address 1
         |            04 = ID Address 2
         |            08 = ID Address 3
         |            10 = ID Address 4
         |            20 = ID Address 5
         |            40 = ID Address 6
         |            80 = ID Address 7
         |
         `---------> ID Address (shown above) of the tape
                     expansion box that is connected to
                     port A.
\end{ttfig}

A status code of 0000.4001 indicates a good status for the tape controller
when no devices are connected to port A. If there is a tape expansion box
connected to port A and the status code is 0000.4001, then the controller
does not recognize it or it is not powered up.

If the second two digits of the status code shown in \figref{2-15} (good
devices) do not match the first two digits (connected devices), then there
is a communication problem between the tape expansion box and the sys-
tem. Disconnect the cable from port A on the expansion adapter and run
self-test again. If the status code for TPC now contains anything other than
0000.4001, replace the system module. Otherwise, if the status code is
0000.4001, reconnect the cable to port A and make sure the tape expan
sion box is powered up. Run self-test again. If the status code for TPC
is not 0202.0001 after reconnecting the cable, run the system exerciser to
thoroughly test the tape controller and the tape expansion box. If the error
code is 0202.0001 after reconnecting the tape expansion box, the tape Con
troller and the expansion tape box are operating properly. Note that there
may be more than one tape expansion box on systems that run an operating
system other than VMS or ULTRIX. This means that the status code may be
0606.0001 for two devices at ID address 1 and 2 or FEFE.0001 for all seven
devices connected to the tape port. However, if the system does use VMS
or ULTRIX, the tape expansion box at ID address 1 is the only device VMS
and ULTRIX can communicate with.

\subsubsection{System Exerciser}

Start the system exerciser by entering TEST 0 for customer mode or install
the 25-pin loopback on the communications port, load the COMPACTape
cartridge with the special-key into the TK50, and enter TEST 101 for field
Service mode. The customer mode system exerciser does not exercise the
tape controller as thoroughly as the field Service exerciser. The field Service
mode system exerciser performs a complete read/write test on the TK50
tape drive and also performs a data transfer test while the customer mode
system exerciser does not test the TK50 tape drive at all.

The system exerciser runs a first pass test on MONO (VAXstation only), DZ,
MEM, and HDC before it runs the first pass test on the TPC. The first pass
test on the TPC tests the tape controller on the system module and checks
the tape port for the presence of a tape expansion box only if the tape Con
troller is operating properly. If the tape controller on the system module
tests bad, the error code for the tape controller lists an error and the tape
port is not checked for the presence of the tape expansion box. The first
pass test of the TPC does not perform any diagnostics or data transfer tests
on the tape expansion box. The first pass test only checks whether or not the
tape expansion box is there and whether or not it can communicate over the
tape port to the tape controller. Complete data transfer and read/write (field
Service mode only) testing is done on subsequent test passes. For exam-
ple, in field Service mode with the Special keyed COMPACTape cartridge
installed, the error code for the tape expansion box shows that the TK50 is
not writeable (1100.0001) during the first pass. However, the second pass
shows that the TK50 is writeable (1300.0001) as long as the cartridge has a
good Special key on it and the tape expansion box is operating properly.
So you must wait for the system exerciser to complete at least two passes
on the TPC to see if the tape expansion box is operating properly.

\figref{2-16} shows the system exerciser display for the TPC. There is one
line for the status of the tape controller on the system module and one line
for each tape expansion box connected to the tape port if it is powered
up. The status of the tape controller is next to the TPC mnemonic (first
line) and the status of the tape expansion box is listed under the status of
the controller. An easy way of determining if any errors are detected is by
looking for the question marks in the left column. If there are any question
marks on the tape controllers status line, the tape controller is faulty and
the error code identifies the problem. Refer to \tabref{2-9} to decifer the tape
controllers error code and determine how to fix the problem. If there are
any question marks on the tape expansion box's status line, the problem
could be in either the tape expansion box, the tape expansion box cable, or
the tape controller on the system module. An error that is listed in the tape
expansion box's error code may be a data transfer error and thus does not
isolate the problem to the tape expansion box; it may still be in the tape
controller on the system module. Refer to \tabref{2-10} to decifer the tape
expansion box's error code and determine how to fix the problem.

If errors are detected with the TPC, disconnect the tape expansion box cable
from port A on the expansion adapter and run the test again. This procedure
isolates the tape expansion box from the system box. If any errors appear
in the tape controllers error code, the problem is with the tape controller
and the system module must be replaced. Otherwise, the error is in the
expansion box and you must run the internal self-test on the expansion box
as described below.

\begin{ttfig}{2-16}{Example of System Exerciser Display for the Tape Controller}
.
.
.
6  00A0    TPC     1000.0001     2     0 00:03:18.26
                   1300.0001
.
.
.
\end{ttfig}

\newpage

\tabref{2-9} lists the system exerciser enor codes for the controller's error line
and \tabref{2-10} lists the system exerciser error codes for the tape drives error
line.

\begin{tbl}{2-9}{TPC Tape Controller System Exerciser Error Codes}{p{0.2\textwidth} p{0.2\textwidth} p{0.5\textwidth}}
\textbf{Error Codes} & \textbf{Possible Cause} & \textbf{Corrective Action}\\
\hline

X000.0001	&	This error code indicates no Controller error. The X indicates
				the transfer test drive number. &
	The X position indicates the tape drive that was used for data transfer testing. This number should
	be 1 if a tape expansion box is connected to the system and it is powered up. This number should be
	8 if no tape expansion box is connected or if it is turned off. An F will always be in this position
	during the first pass of the exerciser. If the P is still in this position after the first pass is complete, a
	fatal error is detected in the tape controller on the system module and the tape expansion box is not
	tested.
\\

0X00.0001	&	The X indicates a data transfer error if X is anything other than zero. &
	Replace the system module. If the error still exists, replace the TK50 tape drive. 
	Then if the error code still exists, replace tbe TZK50 controller board.
\\
00XX.0001	&	The XX indicates the number of data transfer errors detected.
				Make a note of the tape drive used for the data transfer test. &
	Run the exerciser again. Was the same drive used for the transfer test? If yes, and the number of
	transfer errors are the same (or close to the same), replace the system module. If no, and the number
	of transfer errors are zero or considerably less than before, replace the TK50 drive that was originally
	used for the transfer test. If the error code still exists, replace the TZK50 controller board.
\\
0000.XXX1	&	Controller error if any X is anything other than zero. &
	Replace the system module.
\\

\end{tbl}

